\documentclass[12pt,english,a4paper,titlepage,cleardoublepage=empty,dottedtoc]{report}



% % % additions by tompollard START -->

% Overwrite \begin{figure}[htbp] with \begin{figure}[H]
\usepackage{float}
\let\origfigure=\figure
\let\endorigfigure=\endfigure
\renewenvironment{figure}[1][]{%
\origfigure[b]
}{%
\endorigfigure
}

% TP: hack to truncate list of figures/tables.
\usepackage{truncate}
\usepackage{caption}
\usepackage{tocloft}
% TP: end hack

% % % <-- END additions by tompollard



\usepackage{lmodern}
\usepackage{amssymb,amsmath}
\usepackage{ifxetex,ifluatex}
\usepackage{fixltx2e} % provides \textsubscript
\ifnum 0\ifxetex 1\fi\ifluatex 1\fi=0 % if pdftex
  \usepackage[T1]{fontenc}
  \usepackage[utf8]{inputenc}
\else % if luatex or xelatex
  \ifxetex
    \usepackage{mathspec}
  \else
    \usepackage{fontspec}
  \fi
  \defaultfontfeatures{Ligatures=TeX,Scale=MatchLowercase}
\fi
% use upquote if available, for straight quotes in verbatim environments
\IfFileExists{upquote.sty}{\usepackage{upquote}}{}
% use microtype if available
\IfFileExists{microtype.sty}{%
\usepackage{microtype}
\UseMicrotypeSet[protrusion]{basicmath} % disable protrusion for tt fonts
}{}
\usepackage[unicode=true]{hyperref}
\hypersetup{
            pdftitle={Master Thesis: Open Specification of a user-controlled Web Service for Personal Data},
            pdfauthor={G. Jahn},
            pdfkeywords={personal data store; personal data as a service; web service architecture; open specification; masters thesis},
            pdfborder={0 0 0},
            breaklinks=true}
\urlstyle{same}  % don't use monospace font for urls
\ifnum 0\ifxetex 1\fi\ifluatex 1\fi=0 % if pdftex
  \usepackage[shorthands=off,main=english]{babel}
\else
  \usepackage{polyglossia}
  \setmainlanguage[]{english}
\fi
% Make links footnotes instead of hotlinks:
\renewcommand{\href}[2]{#2\footnote{\url{#1}}}
\IfFileExists{parskip.sty}{%
\usepackage{parskip}
}{% else
\setlength{\parindent}{0pt}
\setlength{\parskip}{6pt plus 2pt minus 1pt}
}
\setlength{\emergencystretch}{3em}  % prevent overfull lines
\providecommand{\tightlist}{%
  \setlength{\itemsep}{0pt}\setlength{\parskip}{0pt}}
\setcounter{secnumdepth}{5}
% Redefines (sub)paragraphs to behave more like sections
\ifx\paragraph\undefined\else
\let\oldparagraph\paragraph
\renewcommand{\paragraph}[1]{\oldparagraph{#1}\mbox{}}
\fi
\ifx\subparagraph\undefined\else
\let\oldsubparagraph\subparagraph
\renewcommand{\subparagraph}[1]{\oldsubparagraph{#1}\mbox{}}
\fi

% set default figure placement to htbp
\makeatletter
\def\fps@figure{htbp}
\makeatother



% Table of contents formatting
\renewcommand{\contentsname}{Table of Contents}
\setcounter{tocdepth}{3}

% Headers and page numbering
\usepackage{fancyhdr}
\pagestyle{plain}

% Table package
\usepackage{ctable}% http://ctan.org/pkg/ctable

% Deal with 'LaTeX Error: Too many unprocessed floats.'
\usepackage{morefloats}
% or use \extrafloats{100}
% add some \clearpage

% % Chapter header
% \usepackage{titlesec, blindtext, color}
% \definecolor{gray75}{gray}{0.75}
% \newcommand{\hsp}{\hspace{20pt}}
% \titleformat{\chapter}[hang]{\Huge\bfseries}{\thechapter\hsp\textcolor{gray75}{|}\hsp}{0pt}{\Huge\bfseries}

% % Fonts and typesetting
% \setmainfont[Scale=1.1]{Helvetica}
% \setsansfont[Scale=1.1]{Verdana}

% FONTS
\usepackage{xunicode}
\usepackage{xltxtra}
\defaultfontfeatures{Mapping=tex-text} % converts LaTeX specials (``quotes'' --- dashes etc.) to unicode
% \setromanfont[Scale=1.01,Ligatures={Common},Numbers={OldStyle}]{Palatino}
% \setromanfont[Scale=1.01,Ligatures={Common},Numbers={OldStyle}]{Adobe Caslon Pro}
%Following line controls size of code chunks
% \setmonofont[Scale=0.9]{Monaco}
%Following line controls size of figure legends
% \setsansfont[Scale=1.2]{Optima Regular}

%Attempt to set math size
%First size must match the text size in the document or command will not work
%\DeclareMathSizes{display size}{text size}{script size}{scriptscript size}.
\DeclareMathSizes{12}{13}{7}{7}

% ---- CUSTOM AMPERSAND
% \newcommand{\amper}{{\fontspec[Scale=.95]{Adobe Caslon Pro}\selectfont\itshape\&}}

% HEADINGS
\usepackage{sectsty}
\usepackage[normalem]{ulem}
\sectionfont{\rmfamily\mdseries\Large}
\subsectionfont{\rmfamily\mdseries\scshape\large}
\subsubsectionfont{\rmfamily\bfseries\upshape\large}
% \sectionfont{\rmfamily\mdseries\Large}
% \subsectionfont{\rmfamily\mdseries\scshape\normalsize}
% \subsubsectionfont{\rmfamily\bfseries\upshape\normalsize}

% Set figure legends and captions to be smaller sized sans serif font
\usepackage[font={footnotesize,sf}]{caption}

\usepackage{siunitx}

% Adjust spacing between lines to 1.5
\usepackage{setspace}
% \onehalfspacing
\doublespacing
\raggedbottom

% Set margins
\usepackage[top=1.5in,bottom=1.5in,left=1.5in,right=1.4in]{geometry}
% \setlength\parindent{0.4in} % indent at start of paragraphs (set to 0.3?)
\setlength{\parskip}{9pt}

% Add space between pararaphs
% http://texblog.org/2012/11/07/correctly-typesetting-paragraphs-in-latex/
% \usepackage{parskip}
% \setlength{\parskip}{\baselineskip}

% Set colour of links to black so that they don't show up when printed
\usepackage{hyperref}
\hypersetup{colorlinks=false, linkcolor=black}

% Tables
\usepackage{booktabs}
\usepackage{threeparttable}
\usepackage{array}
\newcolumntype{x}[1]{%
>{\centering\arraybackslash}m{#1}}%

% Allow for long captions and float captions on opposite page of figures
% \usepackage[rightFloats, CaptionBefore]{fltpage}

% Don't let floats cross subsections
% \usepackage[section,subsection]{extraplaceins}


% Chapter styling
% src: https://github.com/chiakaivalya/thesis-markdown-pandoc/blob/master/preamble.tex#L26
\usepackage[grey]{quotchap}
\makeatletter
\renewcommand*{\chapnumfont}{%
  \usefont{T1}{\@defaultcnfont}{b}{n}\fontsize{80}{100}\selectfont% Default: 100/130
  \color{chaptergrey}%
}
\makeatother

\title{Master Thesis: Open Specification of a user-controlled Web Service for
Personal Data}
\author{G. Jahn}
\date{\today}

\begin{document}
\maketitle
\begin{abstract}
Data is the currency of tomorrow. Organizations, whether in the private
or public sector, are gathering enormous amounts of personal (big) data.
This data is harvested and incorporated by these third parties, but were
created by individuals and should, therefore, belong to them. People are
depending on their data. Their identity as well as their personality are
defined by their personal data. Meanwhile data silo operators are
hammering onto these haystacks eagerly trying to find any correlations
worth interpreting, thereby almost inevitably discriminating against the
rightful owners. To reduce the possibility of discrimination only bare
minimum of data required should be handed over to a third party. Thus
the individual has to be in charge of the whole process. A personal data
service will empower its user to regain full control over her data and
facilitates detailed information on every data flow. To be able to trust
such a tool, the user should be able to look inside. Therefore a
personal data service has to be open source and developed transparently,
which would then also encourage self-hosting.
\end{abstract}

{
\setcounter{tocdepth}{1}
\tableofcontents
}
\chapter{Introduction}\label{introduction}

\section{Motivation}\label{motivation}

Nowadays it is rare to find someone that does not collect data about
some kind of thing; particularly humans are the targets of choice for
the \emph{Big Data Movement}
{[}\protect\hyperlink{ref-web_2016_privacy-international-about-big-data}{1}{]}.
Since humans are all individuals, they are - more or less - distinct
from each other. However, subsets of individuals might share a minor set
of attributes, but the bulk is still very unique to an individual, given
that the overall variety of attributes is fairly complex. That small
amount of shared attributes might seem to be less important, due to the
nature of inflationary occurrence, but the opposite turns out to be
true. These similarities allow to determine the individuals who are part
of a subset and the ones who arn't. Stereotypical patterns are applied
to these subsets and thus to all relating individuals. Thus enriched
information are then used to help predicting outcomes of problems or
questions regarding these individuals. In other words, searching for
causation where in best the case one might find correlations - or so
called \emph{discrimination}, which

\begin{quote}
{[}\ldots{}{]} refers to unfair or unequal treatment of people based on
membership to a category or a minority, without regard to individual
merit.
\emph{{[}\protect\hyperlink{ref-paper_2008_discrimination-aware-data-mining}{2}{]}}
\end{quote}

When interacting directly with each other, discrimination of human
beings is still a serious issue in our society, but also when humans
leverage computers and algorithms to uncover formerly unnoticed
information in order to include them in their decision making. For
example when qualifying for a loan, hiring employees, investigating
crimes or renting flats. Approval or denial, the decision is based on
computed data about the individuals in question
{[}\protect\hyperlink{ref-book_2015_ethical-it-innovation_ethical-uses-of-information-and-knowledge}{3}{]},
which is simply discrimination on a much larger scale and with less
effort - almost parenthetically. The described phenomenon is originally
referred to as \emph{Bias in computer systems}
{[}\protect\hyperlink{ref-paper_1996_bias-in-computer-systems}{4}{]}.
What at first seems like machines going rouge on humans, is, in fact,
the \emph{cognitive bias}
{[}\protect\hyperlink{ref-wikipedia_2016_cognitive-bias}{5}{]} of human
nature, modeled in machine executable language and made to reveal the
patterns their creators were looking for - the \emph{``Inheritance of
humanness''}
{[}\protect\hyperlink{ref-web_2016_big-data-is-people}{6}{]} so to say.

In addition to the identity-defining data mentioned above, humans have
the habit to create more and more data on a daily basis - pro-actively
(e.g by writing a tweet) and passively (e.g by allowing the twitter app
accessing their current location while submitting the tweet). As a
result, already tremendous amounts of data keep growing bigger and
bigger, waiting to be harvested, collected, aggregated, analyzed and
finally interpreted. The crux here is, the more data being made
available
{[}\protect\hyperlink{ref-video_2015_big-data-and-deep-learning_discrimination}{7}{]}
to \emph{mine}, the higher the chances to isolate data sets, that differ
from each other but are coherent in themselves. Then it is just a matter
of how to distinguish the data set and thereby the related individuals
from each other.

In order to lower potential discrimination we either need to erase
responsible parts from the machines, thereby it's crucial raising
awareness and teaching people about the issue of discrimination, or we
try to prevent our data from falling into these data silos. The latter
will be addressed in this work.

\section{Purpose \& Outcome}\label{purpose-outcome}

From an individual's perspective providing data to third parties might
not seem harmful at all. Instead eventually one get improved services in
return, e.g.~more adequate recommendations and fitting advertisement, or
more helpful therapies and more secure environments. That said, though
it is a matter of perception what's good and bad, what's harmful and
what's an advantage. Computing data to leverage decision making is
essentially just science and technology and it's up to the humans how
such tools are getting utilized and what purposes they are serving.
Hence it should be decided by the data creators, how their data get
processed and what parts of them are used.

To tackle the described issue the initial idea here is (1) to equip
individuals with the ability to control and maintain their entire data
distribution and (2) thus reducing the amount of \emph{potentially
discriminatory}
{[}\protect\hyperlink{ref-paper_2008_discrimination-aware-data-mining}{2}{]}
attributes leaking into arbitrary calculations. To do so people need a
reliable and trustworthy tool, which assists them in managing all their
\emph{personal data} and making them accessible for 3rd parties but
under their own conditions. After getting permission granted these data
consumers might have the most accurate and reliable one-stop resource to
an individuals's data at hand, while urged to respect their privacy at
the same time. However this also comes with downsides in terms of
security and potential data loss. Elaborating on that and discussing
different solutions will be part of the {[}design
process{]}{[}Design{]}.

The way how to solve the described dilemma is not new. Early days of
work done in this field can be dated back to the Mid-2000s where studies
were made e.g.~about recent developments in the industry or user's
concerning about privacy, and the term \emph{Vendor Relationship
Management (VRM)} were used initially within the context of user-centric
personal data management, which also led into the \emph{ProjectVRM}
{[}\protect\hyperlink{ref-web_2010_projectvrm_about}{8}{]} started by
the \emph{Berkman Klein Center for Internet \& Society at Hardvard
University}. Since then a great amount of effort went into this research
area until today, while also commercial products and business models
trying to solve certain problems. For instance concepts such as the
\emph{Personal Data Store (PDS)}
{[}\protect\hyperlink{ref-paper_2013_the-personal-data-store-approach-to-personal-data-security_2013}{9}{]}
or a \emph{MyData}
{[}\protect\hyperlink{ref-whitepaper_2014_mydata-a-nordic-model-for-human-centered-personal-data-management-and-processing}{10}{]}
implementation called \emph{Meeco}
{[}\protect\hyperlink{ref-web_2016_meeco-how-it-works}{11}{]}, which
will all be covered in a more detailed way within the following chapter.

The work and research done for this thesis will be the foundation for an
\emph{Open Specification}, which by itself is a manual to implement a
concept called \emph{Personal Data as a Service}. Important topics like
how the architecture will look like, where the actual data can be
stored, how to obtain data from the external API or what requirements a
user interface for data management need to satisfy, will be examined.
After the thesis will be finished, the majority of core issues should
already be addressed and can then get outlined in the specification
document. Only then the task to actual implement certain components can
begin. The reason for that is, when sensitive subjects especially like
people's privacy is at risk, all aspects in question deserve a careful
considerations and then get addressed properly. Thus it is indispensable
to put adequate effort primarily into the theoretical work. To be clear
though, that doesn't mean writing code to test out theories and ideas
can't be done during research and specification development. It might
even help to spot some flaws and eventually trigger evolvement.

To ensure a great level of trust to this project and the resulting
software, it is vital to make the development process fully transparent
and encourage people to get involved. Therefore it is required to open
source all related software and documents
{[}\protect\hyperlink{ref-repo_2016_pdaas-spec}{12}{]} from day one on.

In summary, this document is meant to be the initial step in a
development process fabricating a tool to manage all data that defines
her identity controlled and administrated by it's owner, and maybe give
her a more precise understanding about where her personal information
flow and how this might effect her privacy.

\section{Scenarios}\label{scenarios}

The following use cases shall depict different situations and possible
ways such emerging software might be applicable or useful, while
providing it's user with more control over her personal data. Some of
them are more practical and realistic, like ordering and purchasing
online a product, others might have no current usage, but showing a
certain potential to become more relevant when new technologies and
business models emerge, followed by new demands of data.

\begin{itemize}
\tightlist
\item
  order sth online, purchase, and package shipment
\item
  social network accessing arbitrary profile data
\item
  credibility (applying for a loan) validation by a certain financial
  institution: accessing arbitrary data
\item
  patient/health record
\item
  care (movement) data
\end{itemize}

\subsubsection{Ordering a product
online}\label{ordering-a-product-online}

The data owner searches through the web to find a new toaster, since her
old one recently broke. After some clicks and reviews, she found her
soon-to-become latest member of the household's kitchenware. After
putting the model name in a price search engine, hoping to save some
money, the first entry, offering a 23\% discount, caught her attention.
She decides to have a deeper look into the toasters and thus has heading
towards the original web shop entry. Finally she came around and put the
item onto her card, despite tha fact, that she has never bought
something from that online shop before. Then she proceeded to checkout
to place her order. The shop-interface is asking her to either insert
her credentials, proceed without registration or sign-in, or insert s
URI to an endpoint of her \emph{Personal Data as a Service}. TODO: the
following description might need some adjustments according data flow /
process description She opens up the management panel of her
\emph{PDaaS} and creates a new entry in a list of data consumers, that
already have access to characteristics of her personal data. As a
result, she receives a URI, which she inserts according, as mentioned
before; after she assures herself that the data exchange with the shop
through the browser is based on a secure connection (HTTPS). Under this
URI, the shop-system can then request data, that is required for a
successful transaction. Moving on to the next step after submitting the
URI, the data owner is ask to decide how she would like to pay. The
choices are: credit card, invoice, paypal or bank transfer. She chooses
the last one, submits her selection and thereby completes her order.
After a moment, a push notification pops up on her mobile device, which
is a permission request from her \emph{PDaaS}, asking for granting the
shop-system, she just places the order, access to her full name, address
and email. Additionally she can decide between three states of how log
the permission wil be valid: \emph{one-time-only},
\emph{expires-on-date} and \emph{until-further-notice}. Since she never
ordered at this shop before and might never again, she decided to grant
access only for this specific occasion. After the shop-system receives
the data, it sends an email to the data owner, containing some
information about her order, including the shop's bank details. which
then enables her to actually pay the amount due. After the system
recognizes the payment has coming in, it triggers the shipment of the
toaster. In order to get a full impression of how the whole process
might have look like when the data owner had chosen one of the other
payment methods, the differences will be describes in the following. If
the data owner would have wanted to pay with her credit card, the only
difference would have been, that the shop-system had requested also to
access the credit card number and it's belonging secret, and when
sending the email the system would have omitted the information about
the shop's bank details. Being able to choose paying with invoice where
possible only because the \emph{PDaaS} response has indicated, that it's
containing \emph{profile data} is certified and therefore trustworthy.
Which reduces the shop owner's risk and would have enabled him in case
of fraud or misuse to take action. Choosing to involve paypal as a
\emph{middleman} to process the payment, requires the data owner to had
already granted paypal certain access to her \emph{PDaaS}. If that's the
case, then the shop-system would have ask also for her paypal-ID, which
then the system will use to request the payment directly from paypal.
This on the other side will cause paypal to consult the \emph{PDaaS},
which results in a second notification, asking the data owner for
permission to proceed. After the payment transfer was successful, the
shipment will gets initiated. And with the package arriving at the data
owner's doorstep the whole transaction has finished.

\subsubsection{Interacting with a social
network}\label{interacting-with-a-social-network}

Entering a social network for the first time, only take the URI to the
data owner's \emph{PDaaS} and a password. The data owner receives a
notification on her mobile device asking for permission to access
certain data about her. If her mobile device is currently not at hand,
she can also use the administration panel provided by her \emph{PDaaS}
and reachable with a web browser on every internet-enabled device.
Within that panel pending permission reviews will be indicated. Whether
the data owner has already reviews the request or not, she should be
able to login to the social network. After doing so, she should not be
able to see any of her information. After granting permissions to the
social network to accessing certain data \emph{until-further-notice} and
reloading the session, she then should see all her So every time,
someone on that network tries to access her information, whom she has
allowed to see that information (which is managed by the user only from
within the network), the network pulls the data from the owner's
\emph{PDaaS}, if it's still permitted to do so. It's also imaginable,
that the social network and a \emph{PDaaS} are establishing a backward
channel. This channel could be used to send all the content she would
create over time while interacting with the social network and it's
participants back to her \emph{PDaaS}. The network itself only stores a
reference to all content object, whether it's for example an image, a
post or comment on somebody else's post and if it's needed the actual
content will be fetched from the owner's \emph{PDaaS}.

\subsubsection{Applying for a loan and checking
creditworthiness}\label{applying-for-a-loan-and-checking-creditworthiness}

The data owner would like to buy an apartment. In order to finance such
a acquisition, she needs a funding, which in her case, will be based on
a loan. During a conversation in a credit institute of her choice, an
account consultant describes to her what data will be required in order
to decide about her creditworthiness. While giving a consensual nod, she
takes out her smartphone and brings up the management panel of her
\emph{PDaaS}. With a few taps she has just created a new \emph{data
consumer}. The panel then shows a QR-Code, that holds a URI to a
dedicated endpoint of the data owners \emph{PDaaS}. She shows that code
to her consultant, who then scans it. While handling some more
formalities and talking about several issues and possible products she
might be interested it, she gets a notification on her phone, informing
her about a permission request the institute just made. It lists all the
different data points the institute would like to access in order to
calculate her scoring, such as address, monthly income, relationship
status and family, history of banking or other current loans. After some
back and forth and solving some misunderstandings with the help of her
consultant, she decided to just partially allow access to the requested
data and just for this time and purpose. The consultant kindly pointed
out, that these decisions might have an impact on the scoring and
thereby on the lending and it's terms. After the consultant got a signal
from the computer system, the two then finishing up their meeting and
the consultants informed the data data owner about the next steps, which
includes a note, that the institute will contact her within the next few
days, when they have come to a conclusion. In case of a positive outcome
a new appointment need to be made, for doing all the paperwork and
signing the contract. From a technical point of view, two different ways
of computing the score are imaginable. The first one would be,
transferring only the plain data - request, containing the query and
response containing the data - including the expire date and information
regarding the signature state. But the actual computations and analytics
to obtain the score, will happen within the infrastructure of the credit
institute. When this process is over, all transferred personal data has
to be deleted. An alternative could prevent the data from leaving the
\emph{PDaaS}, in which the institute's request won't consists of a data
query. Instead it would came along with an chunk of software and some
information on how to run it. The \emph{PDaaS} server will provide an
isolated runtime in which the software then gets executed. After the
process has finished, the result will be send back to the credit
institute's infrastructure.

\subsubsection{Maintain and provide it's own health/patient
record}\label{maintain-and-provide-its-own-healthpatient-record}

Some time ago on a hiking trip in a moment of carelessness the data
owner has accidently broke her leg. She came into a hospital and went
straight into surgery, where the physicians could fix the injury. Time
went by and the leg has healed completely. After she woke up today she
felt some pain coming from that area where her leg was broken. She
decided to call in sick and went straight to a doctor nearby. During her
recovery she visited that doctor regularly. At the reception desk, she
opens up the \emph{PDaaS}'s management panel on her smartphone and
searched through the list of data consumers. After she found the entry
for this clinic, she flipped her phone to show the receptionist the
corresponding QR-Code, which she started to scans immediately. However
the receptionist couldn'd see any data on the screen, because the access
has already expired. The data owner only had permitted access for the
estimated time of recovery, which was over some time ago. That's why she
got a notification, to re-grant some access. Going through the data
points the clinic-system has requested, she noticed that her address is
incorrect. Last month she moved out and into a bigger apartment just
down the street. She must have forgot to change that data, which she
corrects immediately right before submitting the access configurations
for the clinic-system. She also included the access to all the data
originated from that time after her accident. A moment later the
receptionist confirms to now being able to see all necessary data. The
data owner takes a seat in the waiting room. While passing some time,
she had a deeper look into her list of data consumers; some of them she
couldn't even remember and for others she was surprised to what data she
has granted access to and started to reduce certain permissions, if it
was appropriate in her eyes. She even removed some of the entries. The
appointment with her doctor went great. He even had to review the x-ray
images in order to make a adequate differential diagnosis. After the
visit, she had to make a quick stop at a pharmacy along the way to
pickup the drugs her doctor had prescribed for her to reduce the pain.
She had to wait in the queue with two other customers being in front of
her. She realized, that it's the first time she has been here. So she
prepared a new entry in her data consumer list, including all
information about her prescriptions. So by the time she get served, she
just let the person behind the register scan her code. In the next
seconds the data owner gets a quick confirmation notification about the
request that just happened. A moment later the pharmacist come back with
her drugs, which she then pays in cash and the transaction is done.

\subsubsection{Vehicle data and
mobility}\label{vehicle-data-and-mobility}

Assuming a car itself has no hardware on board in order to establish a
wireless wide area connection to an outside access node. Only from the
inside one can connect to the car (wired or wireless). After entering a
car, on the data owner's mobile device pops up a notification asking for
permission to connect to that device. In addition to the expiration
date, the data owner can choose to en- or disable two more options.
First, a wifi network with an uplink to the internet can be provided to
everyone inside the car. Secondly, connections, the car might want to
establish, in order to emit data via internet - which, regardless, have
to go through the currently linked mobile device. Thus the device owner
gains full control over any external data transfer that might happen.
This again would allow two things: (A) permission management for all
outgoing data and (B) funnel all data generated and provided by the car
into the \emph{PDaaS} associated with that linked device. It might also
be feasible to deny any connection the car is trying to make. Thus the
data will only be stored in the \emph{PDaaS}. If somebody is interested
in such then have to ask for access permission. That same concept about
movement tracking and vehicle data could also be applied to driving
(motor) bicycle.

\section{Terminologies}\label{terminologies}

\textbf{Web Service:} TODO

\textbf{Open Specification:} TODO

\textbf{Big Data:} deep learning, neural networks

\textbf{profile data:} individual's inherent data; TODO

\textbf{Personal Data:} TODO

\textbf{Personal Information} predominantly static data points related
to an individual

\textbf{Personal Data as a Service (PDaaS):} a web service controlled,
owned and maybe even hosted by an individual, which provides access to
the owner's personal data and offers maintainability as well as
permission management.

\textbf{Personal Data Store:} TODO

\textbf{Vendor Relationship Manager:}
{[}\protect\hyperlink{ref-web_2010_projectvrm-wiki_about-vrm}{13}{]}

\textbf{Personal Information Management Systems (PIMS):}
{[}\protect\hyperlink{ref-web_2010_projectvrm-wiki_pims-example-list}{14}{]}

\textbf{serverless:} TODO
https://auth0.com/blog/2016/06/09/what-is-serverless/

\textbf{Digital Footprints:} TODO

\textbf{Owner:} person who controls (and probably hosts) the data
service containing her personal data (TODO: or maybe \emph{data
source}/\emph{subject} and \emph{data collector})

\textbf{(Data) Consumer:} Third party, external entity requesting data,
authorized by the owner to do so

\textbf{Data Broker(s):} entities with commercial interests, that
collect, aggregate and analyze information/data of any kind - in this
case about human beings - from different sources in order to enrich the
data sets, to finally license the resulting corpora to other
organisations.
{[}\protect\hyperlink{ref-report_2014_data-brokers}{15}{]}

\textbf{permission request:} fist attempt to request access to certain
data in the \emph{PDaaS}

\textbf{access profile:} a data set about a third party that already
made an permission request. The set contains additional information and
access rules

\textbf{data access:} after a third party's \emph{permission request}
got reviewed and saved, that entity is then able to make an attempt to
access data.

\chapter{Fundamentals}\label{fundamentals}

The following chapter shall provide the foundational knowledge about
concepts like \emph{Personal Identity} or \emph{Big Data} and therefore
ensures a common understanding on their relation to the problem this
work tries to solve. Additionally it is given a brief overview on what
existing standards and technologies might be used, and summarizes the
research already been made as well as it's current state.

\section{Digital Identity, Personal Data and
Ownership}\label{digital-identity-personal-data-and-ownership}

\begin{itemize}
\tightlist
\item
  \emph{Digital Identity}

  \begin{itemize}
  \tightlist
  \item
    what is a \emph{DI}? and in comparison to \emph{Personal Data}?
  \item
    what is required to make the PDaaS used or seen as a \emph{DI}?
  \end{itemize}
\item
  \emph{Personal Data} definition

  \begin{itemize}
  \tightlist
  \item
    general - freely spoken
  \item
    as of EU law (incl citation)
  \item
    as of US law (incl citation)
  \item
    is it just policy/guideline or enforceable too (law/rule)? what
    relevance/impact have companies \emph{terms and conditions}?
  \item
    EU and USA (since server might be located outside the state or
    effective range)
  \end{itemize}
\item
  \emph{Ownership} of personal data

  \begin{itemize}
  \tightlist
  \item
    who is the owner in what situation or under what circumstances?
  \item
    am I the owner when I was the one who was collecting them? Does it
    depend on whether the resource was public or somewhat private?
  \item
    what will happen with her data service after a person died?
  \end{itemize}
\end{itemize}

\begin{itemize}
\item
  A \textbf{Digital Identity} is a non-physical abstraction of an
  entity, such as an organisation, an individual, a device or even
  software, which allows bidirectional association. In the context of
  this document, it only refers to human beings. Therefore a
  \emph{digital identity} is the individual's representation in digital
  systems, consisting of identity-defining data, such as \emph{personal
  information} and it's history and preferences
  {[}\protect\hyperlink{ref-whitepaper_2012_the-value-of-our-digital-identity_definition}{16}{]}.
  \emph{Personal information}, in this case, refers to inherent (date of
  birth) and imposed (credit card number) characteristics.
\item
  From a technical perspective a DI is essentially a collection of
  characteristics, attributes and time series data (e.g.~interaction
  logs or bank account history). A subset of these attributes combined
  can form unique fingerprint, like certain single data points
  (e.g.~social security number) in their own context might be, too. Thus
  it might not be necessary to know the values of all attributes in
  order to identify a person as the rightful owner and physical
  counterpart. It can also be seen as an avatar in the digital world or
  as the digital part of a human's identity. Therefore its important to
  not view the \emph{DI} as a reduction of a living individual to some
  bits and bytes, but rather as a appropriate representation for certain
  purposes and contexts.
\item
  It is also possible to provide an additional level of authenticity
  insurance for data related to an entity. Therefor an unrelated third
  party, which needs to be approved not only by the related individual,
  but also by all entities participating in a context, which might be
  relevant e.g.~for some administration purposes.
\item
  But the concept would also impose a new level of attacking vectors to
  the identity owner, such as identity theft. The attacker is no longer
  required to be physically present to be able to steal certain unique
  identifiers from a person. It is sufficient to gain access to area
  where the sensitive data is stored.
\item
  In the context of this document and all related work, \textbf{Personal
  Data} is specified as a combination of an individual's \emph{Digital
  Identity} and all of it's ever created intellectual property
  (e.g.~posts, images, tweets or comments). This includes all sorts of
  tracking data and interaction monitoring, as well as metadata manually
  or automated enriching content (e.g.geo-location attached to a tweet
  as meta information). Data, captured by someone ore something on or
  about the individual's private living space and property. Simply every
  data point reflecting the individual's personality - partly or as a
  whole - is seen as \emph{personal data}.
\item
  The european \emph{Data Protection Regulations} defining
  \emph{Personal Data} as follows: \textgreater{} `personal data' means
  any information relating to an identified or identifiable natural
  person \textgreater{} (`data subject'); an identifiable natural person
  is one who can be identified, directly or \textgreater{} indirectly,
  in particular by reference to an identifier such as a name, an
  identification \textgreater{} number, location data, an online
  identifier or to one or more factors specific to the physical,
  \textgreater{} physiological, genetic, mental, economic, cultural or
  social identity of that natural person; \textgreater{}
  \emph{{[}\protect\hyperlink{ref-regulation_2016_eu_general-data-protection-regulation_definition}{17}{]}}
\item
  The U.S.A. has little legislation on defining and protecting
  consumer's privacy. At least they have no explicit bills addressing
  such area
  {[}\protect\hyperlink{ref-web_2016_wikipedia_information-privacy-law_us}{18}{]}.
  Though some of the existing sectoral laws consist of partially
  applicable policies and guidelines
  {[}\protect\hyperlink{ref-web_2016_data-protection-laws-in-the-us}{19}{]};
  most of them addressing specific types of data. In 2015 the White
  House made an attempt to fill the gap with the \emph{Consumer Privacy
  Bill of Rights Act}, but to this date it didn't passes the draft
  state. According to the critics, it lags of concrete enforceable rules
  consumers can rely on
  {[}\protect\hyperlink{ref-web_2015_white-house-releases-consumer-privacy-bill-draft}{20}{]}.
  The draft contains a general definition of \emph{Personal Data}:
  \textgreater{} ``Personal data'' means any data that are under the
  control of a covered entity, not otherwise \textgreater{} generally
  available to the public through lawful means, and are linked, or as a
  practical matter \textgreater{} linkable by the covered entity, to a
  specific individual, or linked to a device that is \textgreater{}
  associated with or routinely used by an individual, including but not
  limited to {[}\ldots{}{]} \textgreater{}
  \emph{{[}\protect\hyperlink{ref-bill-draft_2015_us_consumer-privacy-bill-of-rights-act_definition}{21}{]}}
\item
  followed by a list of concrete data points, e.g.~email or postal
  address, name, social security number and alike. Aside from the
  legislation with bills, a few third-party organisation can also
  participate by and add new or overwriting existing rules and policies.
  Namely for example the \emph{Federal Communications Commission} (FCC),
  recently releasing \emph{Rules to Protect Broadband Consumer Privacy}
  including a list of categories of sensitive information
  {[}\protect\hyperlink{ref-rules_2016_fcc_to-protect-broadband-consumer-privacy_sensitive-types-of-data}{22}{]},
  which wants \emph{Personally Identifiable Information} (alias Personal
  Data) to be understood as: \textgreater{} {[}\ldots{}{]} any
  information that is linked or linkable to an individual.
  {[}\ldots{}{]} information is \textgreater{} ``linked'' or
  ``linkable'' to an individual if it can be used on its own, in
  context, or in \textgreater{} combination to identify an individual or
  to logically associate with other information about a \textgreater{}
  specific individual. \textgreater{}
  \emph{{[}\protect\hyperlink{ref-rules_2016_fcc_to-protect-broadband-consumer-privacy_personally-identifiable-information}{23}{]}}
\item
  Despite minor difference in detail, they all have similar ideas of
  personal data and their belonging. Even though, the version proposed
  by EU is almost identical with the definition introduced for the
  context of this work. Although the FCC's statutory authorities might
  be somewhat debatable regarding certain topics, the
  \emph{Communications Act} as a U.S. federal law equips the FCC with
  power to regulate and legislate.
\item
  Having a common opinion on what data points are belonging to person is
  the foundation to define a set of rules on how deal with
  \emph{Personal Data} accordingly. Every business, operating within the
  EU, is required\footnote{according to article 12-14 of the \emph{EU
    General Data Protection Regulation 2016/679}} to provide it's users
  with a \emph{Privacy Policy}, while e.g.~in the U.S. - as mentioned
  above - only partially and depending on context and data type users
  must be informed about which and how their data get processed
  {[}\protect\hyperlink{ref-web_2016_privacy-policies-are-mandatory-by-law}{24}{]}.
\item
  A user commonly agrees on the privacy policy, by starting to interact
  with the author's business, thus every \emph{Privacy Policy} is
  required to be publicly accessible; e.g. before creating an account.
  \textgreater{} By clicking Create an account, you agree to our
  \href{https://www.facebook.com/legal/terms}{Terms} \textgreater{} and
  that you have read our
  \href{https://www.facebook.com/about/privacy}{Data Policy}, including
  \textgreater{} our
  \href{https://www.facebook.com/policies/cookies/}{Cookie Use}.
  \textgreater{}
  \emph{{[}web\_2016\_facebooks-landing-page\_policy-acknowledgement{]}}
\item
  It can be seen more likely an information notice, that translates and
  specifies general given law, rather then a contract.
\item
  With such knowledge at hand, it is up to each individual, if the
  service's benefits are worth sharing some personal data, while
  simultaneously acquiescing potential downsides concerning the privacy
  of such data.
\item
  Every entity who is doing so, muss process Personal data according to
  the law and their \emph{Privacy Policy}. If they policies are
  violating existing law or the entity effectively goes against the law
  with their actual doing, penalties might follow. Depending on the
  level and impact of their infringement in addition the law itself,
  aside from revising their wrong-doings the entity might have to
  compensate the affected individuals, pay a fine or get revoked their
  license.
\item
  Not only privacy laws, but every legal jurisdiction has it's
  limitations - concerning their territorial nature - which makes
  legislation not exactly an appropriate tool when it comes to fixing
  existing issues and strengthen the individual's privacy and rights in
  a global context like the \emph{world wide web}. If no international
  agreement is in place
  {[}\protect\hyperlink{ref-web_2016_international-privacy-standards}{25}{]},
  only those laws are considered valid and enforcible where the
  organisation is registered, and maybe the fact where (meaning in which
  area of jurisdiction) the their servers are located or the data is
  processed and stored.
\end{itemize}

Whereas \textbf{Ownership} of \emph{Personal Data} has no legal ground
foundation so ever. The concepts of intellectual property protection and
copyright might intuitively be applicable, because the data, created by
its owner, seems to be her \emph{intellectual property}. Such property
implies to be a result of a creative process though, but unfortunately
there is no \emph{threshold of originality} in facts, like
\emph{personal information} is
{[}\protect\hyperlink{ref-paper_2014_who-owns-yours-data}{26}{]}.

\begin{itemize}
\item
  Ownership in the sense of having exclusive control over it's personal
  data and how they get processed at any given point in time; this not
  only comes with high costs, but is also very inconvenient for both
  parties - owner and data consumer. It consists of two concepts: (A)
  the right to do what every is desired with their property and (B) in
  which rules and mechanisms the ownership can be assigned to someone
  {[}\protect\hyperlink{ref-book_1987_private-ownership_definition}{27}{]}.
\item
  The european DPR\footnote{EU Data Protection Regulation} contains only
  one occurrence of the word \emph{ownership}, which is not even related
  to the context of \emph{personal data} or the \emph{data subject}. It
  only stats, that \emph{``Natural persons should have control of their
  own personal data.''}
  {[}\protect\hyperlink{ref-regulation_2016_eu_general-data-protection-regulation_ownership}{28}{]}.
  Whereas Commissioner J. Rosenworcel of the FCC wants \emph{``consumers
  {[}\ldots{}{]} to {[}\ldots{}{]} take some ownership of what is done
  with their personal information.''}
  {[}\protect\hyperlink{ref-rules_2016_fcc_to-protect-broadband-consumer-privacy_ownership}{29}{]}
\item
  Typically the question of data ownership is addressed in data
  consumer's \emph{Terms of Service} (ToS), which an individual might
  have to accept in order to establish a (legal) relationship with it's
  author. I should be kept in mind, that \emph{ToS} might change over
  time; not necessarily to the users advantage. All addressed issues (by
  the ToS) must not violate any applicable or related law, otherwise the
  \emph{ToS} might not be legally recognized. Taking the following
  excerpts from different \emph{ToS}:
\end{itemize}

\begin{quote}
You own all of the content and information you post on Facebook, and you
can control how it is shared {[}\ldots{}{]}. \emph{(under ``2. Sharing
Your Content and Information'', by Facebook
{[}\protect\hyperlink{ref-web_2016_facebook_terms-of-service}{30}{]})}
\end{quote}

\begin{quote}
You retain your rights to any Content you submit, post or display on or
through the Services. What's yours is yours --- you own your Content.
\emph{(under ``3. Content on the Services'', by Twitter
{[}\protect\hyperlink{ref-web_2016_twitter_terms-of-service}{31}{]})}
\end{quote}

\begin{quote}
Some of our Services allow you to upload, submit, store, send or receive
content. You retain ownership of any intellectual property rights that
you hold in that content. In short, what belongs to you stays yours.
\emph{(under ``Your Content in our Services'', by Google
{[}\protect\hyperlink{ref-web_2016_google_terms-of-service}{32}{]})}
\end{quote}

\begin{quote}
Except for material we may license to you, Apple does not claim
ownership of the materials and/or Content you submit or make available
on the Service ``(under''H. Content Submitted or Made Available by You
on the Service``, by Apple
{[}\protect\hyperlink{ref-web_2016_apple-icloud_terms-of-service}{33}{]})*
\end{quote}

All these statements are followed by the same term, stating that the
user grants the author a worldwide license to do almost any imaginable
thing with her data. This even applies to Apple, if the user is
\emph{``submitting or posting {[}\ldots{}{]} Content on areas of the
Service that are accessible by the public or other users with whom
{[}the user{]} consent to share {[}\ldots{}{]} Content''}
{[}\protect\hyperlink{ref-web_2016_apple-icloud_terms-of-service}{33}{]}.

\begin{itemize}
\item
  It is worth noticing, that in every \emph{ToS} it is only referred to
  the data owner's content, not all her personal data. As mentioned
  above, personal information are no intellectual property, but playing
  an important role in data analytics though. Which is why \emph{privacy
  policies} are in place, to ensure at least some user enlightenment,
  even though it doesn't compensate the lack of control.
\item
  In addition to that, the meaning of \emph{ownership} used in the
  quoted \emph{ToS} is missing a clear outline and thus causing
  ambiguity and leaving room for interpretation. Nor the actual
  definition of \emph{ownership}, as described earlier, is applicable
  for these kind of cases, since the user losing all its control is by
  design. Handing over data to the consumer annihilates the exclusive
  control over the data and revokes the ability of assigning such
  control. There is no (legislation based) way to establish a feasible
  concept of \emph{ownership}, if the data consumer has no motivation to
  promote the user a comprehensive owner of her data.
\item
  Leaving all the legal layer aside for a moment and switching the
  perspectives a bit; Data consumers might argue, that they had invested
  in enabling themselves to collect, process and store personal data, so
  it belongs to them. But from the data owner's point of view it might
  only be the case as long as as she would benefit as well somehow,
  e.g.~using products, services or features, offered by consumers, which
  quality depends on personal data. If the data owner chooses to move to
  a competitor might what to bring her personal data with her. But then
  again the former data consumer would object, competitors would benefit
  from all investments the consumer has made, but without any effort.
  Though, not entirely wrong, two aspects need to be emphasize. (A) In
  order to archive a high level of quality for their analytics and
  therefore in making right decisions to gain improvement, it's vital to
  huge amount of effort in developing these underlying technologies, not
  only in acquiring personal data. Which again only constitutes (B) the
  foundation of various subsequential computations followed by an
  ongoing collecting, aggregation and analytics of actively and
  passively created data and metadata (e.g.~food deliver history or
  platform interactions and tracking). Given the initially introduced
  definition of \emph{personal data} only a fraction of the involved
  data belongs to its owner. The large part consists of highly valuable
  metadata
  {[}\protect\hyperlink{ref-web_2013_why-metadata-matters}{34}{]}
  {[}\protect\hyperlink{ref-web_2016_why-you-need-metadata-for-big-data-to-success}{35}{]}
  and therefore should remain to the data collector and either be
  deleted or sufficiently anonymized, if the owner cancels the
  relationship. The data owner should not depend on the collector's
  willingness when it comes to handing over her personal data (e.g.~list
  of favorites or delivery history). Instead, using her own tool to
  provide the consumer with required data (e.g.~list of favorites) or
  tap into her data creating interactions (e.g.~food deliveries) on her
  own.
\item
  Whether an individual dies or a user deletes her account, as long as
  certain data point are shared with / connected to other users, the
  data will remain. At least when it comes to facebook.
\item
  Generally speaking, all data solely associating with an individual, is
  in the ownership of the same. But since it doesn't exist any legal
  concepts on \emph{personal data} ownership, a technical solution could
  help to regain some control.
\end{itemize}

\section{Personal Data in the context of the Big Data
Movement}\label{personal-data-in-the-context-of-the-big-data-movement}

\begin{itemize}
\item
  big data itself initially can be seen as a \emph{huge blob of data}
  containing more or less structured data sets
  {[}\protect\hyperlink{ref-web_2016_oxford_definition_big-data}{36}{]},
  whose size might have exceeded the capabilities of retrieving certain
  information almost only by hand. Such high data haystacks usually come
  along with new challenges in logistic and resource management, when
  information retrieval needs to get automated on a large scale
  {[}\protect\hyperlink{ref-web_2016_wikipedia_definition_big-data}{37}{]}.
  Theses practices are commonly referred to \emph{Big Data (Analysis)}
  including distributed computing and machine learning.
\item
  Big Data, or to be more precise, collecting and analyzing big data,
  serves the prior purpose to extract useful information, which on the
  other hand depends on what was the opening question about, but also
  what data sets the corpus is containing.
\item
  At first, (A) formalizing question(s) that the results have to answer.
  Secondly, (B) deciding what data is needed and appropriate and then
  start collecting. Third, (C) designing data models accordingly and
  correlate with the data (D) next, analyse and interpret the results.
  (E) last but not least, make business decisions based und the analyses
  ({[}\protect\hyperlink{ref-paper_2015_big-data-analytics_a-survey}{38}{]}
  Fig. 3).
\item
  since quite a few businesses (in terms of purpose or intention) are
  based around the concept of customers, which are generally somewhat
  entities consisting of at least one human being, personal data takes a
  major part in what \emph{Big Data} can be about. In the context of
  this thesis, these entities are individuals with a unique identity.
  And to understand the behaviour, decision making and needs of her
  customers a vendor, who owns the business, needs to know as much as
  possible about them, when she wants to know what changes she needs to
  address in order to move towards the most lucrative business.
\item
  personal data and information are reflecting all this knowledge. It
  starts with profile data, such as gender, age, residency or income,
  goes on with time series events like geo-location changes, or web
  search history and goes all the way up to health data and self-created
  content like \emph{Tweets}\footnote{public massages published by an
    account on \url{twitter.com}, which will be displayed in the
    timeline of all her subscribers and also might contain additional
    types of content like images, links or video} or videos.
\item
  all these classes of personal data hold a major share\footnote{it
    doesn't matter whether an individual or just someone on behalf of an
    organisation spend money for something. at the end of the day, they
    are all humans on this planet and in a capitalistic oriented world
    money needs to flow and profits needs to be maximized. So to know
    where it will flow or why it will flow in a certain direction it is
    crucial to know everything about it's decision maker - the humans on
    this planet.} in the field of data analytics (TODO: find statistics
  showing shares of data types/classes/categories,
  {[}\protect\hyperlink{ref-book-chapter_1999_Principles-of-knowledge-discovery-in-databases_introduction-to-data-mining}{39}{]}
  {[}\protect\hyperlink{ref-web_2013_big-data-collection-collides-with-privacy-concerns}{40}{]})
\item
  but, depending on the specific attributes, they might be not that easy
  to acquire. in general most businesses obtain data from within their
  own platforms. some data might be in the user's rang of control
  (e.g.~customer or profile data), but most of the data comes from
  interacting directly (content creation, inputs) or indirectly
  (transactions, meta information). the level of sensitivity is mainly
  based on the purpose of the platform (benefit for the user) and what
  is the provider's demand from the users commitment (e.g.~required
  inputs or usage requires access to location)
\item
  from a technical perspective collecting passively created data is as
  simple as integrating logging mechanisms in the program logic. since
  the industry moved towards the cloud\footnote{side note - one might
    come to the conclusion, that only the trend towards the \emph{cloud}
    made it actually possible to collect to such an extent we are all
    observing these days, because standalone software should not
    necessarily require internet connection and therefore the vendors
    had no way to gather information whatsoever} most scenarios utilized
  server-client architectures. Furthermore the \emph{always-on}
  philosophy evolved to an imperative state. standalone software is
  starting to call the author's servers from time to time, just to make
  sure the user behaves properly. For browsers it was already a common
  narrative to make here and then requests to the server - still
  preventable though, but when it comes to native mobile apps it is
  almost impossible
  {[}\protect\hyperlink{ref-web_2016_answers-io}{41}{]} to notice such
  behaviour and therefore preventing apps from doing so.
\item
  these architectural developments were inducing the gathering of
  potentially useful information from all over the system on a large
  scale
  {[}\protect\hyperlink{ref-web_2016_big-data-enthusiasts-should-not-ignore}{42}{]}.
  Logging events, caused by the user's interactions, on the client,
  which then get forwarded to backend servers. Or keeping track of all
  kinds of transactions, which is done directly in the backend. Before
  running together in a designated place, all these collected chucks of
  data (TODO or ``data points'') are getting enriched with meta
  information. Finally get stored and probably never removed again - all
  for later analyses.
\item
  The mindset in the \emph{Big Data Community} is grounded on the basic
  assumption of \emph{more data is more helpful}, which already is
  emphasised by the often-cited concept of the three \emph{Vs} (Volume,
  Velocity, Variety)
  {[}\protect\hyperlink{ref-report_2001_3d-data-management-controlling-data-volume-velocity-and-variety}{43}{]}.
  which is not entirely wrong, because it lies in the nature of pattern
  and correlation discovery, to provide increasing quality results
  {[}\protect\hyperlink{ref-paper_2015_big-data-for-development-a-review-of-promises-and-challenges:more-data}{44}{]},
  while enriching the overall data with more precise data sets. But when
  new technologies are emerging, questioning the downsides and possible
  negative mid- or long-term impacts are typically not very likely to be
  a high priority. The focus lies on e.g.~trying to to reach and
  eventually breach boundaries while beginning to evolve. So
  non-technical aspects such as privacy and security awareness doesn't
  come in naturally, instead a wider range of research needs to be done
  alongside the evolution process and the increasing adoption rate in
  order to uncover such issues. Only then they can addressed properly on
  different levels - technical, political as well as social. So that the
  \emph{Big Data Community} itself is able to evolve, too. All in all
  it's a balancing act between respecting the user's privacy and having
  enough data at hand to satisfy the initial questioning with the
  computed results. Therefore people working in such contexts need to
  have advanced domain knowledge, be aware of any downsides or pitfalls
  and need to be sensible about the ramifications of their approaches
  and doings. Such improvements are already happening, not only
  originating from the field's forward thinkers
  {[}\protect\hyperlink{ref-web_2016_the-state-of-big-data}{45}{]}, but
  also advocated by governments, consumer rights organisations and even
  leading Tech-Companies start trying to do better
  {[}\protect\hyperlink{ref-web_2016_apple_customer-letter}{46}{]}
  {[}\protect\hyperlink{ref-web_2016_what-is-differential-privacy}{47}{]}
  {[}\protect\hyperlink{ref-web_2016_eff_whatsapp-rolls-out-emd-to-end-encryption}{48}{]}
  - as discussed in the section {[}TODO see personal data as of the
  law{]},
\item
  earlier in the text a difference was made between actively created and
  passively created data
\item
  based on that one could say \emph{profile/account data} is actively
  created, because it got into the system by the user's actively made
  decision to insert these information into a form and submit it - for
  whatever reason. whereas detecting the user's current location and
  adding this information to the submitted form is \emph{meta data}
\item
  of cause, it is debatable whether these kind of data belongs, in the
  sense of being the rightful owner, to the user or to the author or
  owner of the software containing the code that effectively created the
  data.
\item
  maybe personal data is every data/information whose creation (or
  digital existence) is a direct result of user interaction/engagement?
\item
  lets have a look into what the rule book says about that
  --\textgreater{} next topic (law)
\end{itemize}

\section{Personal Data as a Product}\label{personal-data-as-a-product}

\begin{itemize}
\item
  \emph{Big Data Analytics} by itself just comprises a structured and
  technical-aided procedure, serving the purpose of finding invisible
  information, that might be helpful to make (right) (business)
  decisions. Though, if one would ask data collectors about their
  motivation, most likely the answer would be something along the lines
  of PR phrasing like \emph{``We want to have a better understanding of
  our customers''}. But to do what exactly? To predict what might be the
  next thing I am supposed to buy Or what things I probably would like
  to consume but most certainly not yet know of?
\item
  Let's take a look at some examples. An advertising service uses
  tracking data for targeted advertising. The more information they have
  about an individual, the more accurate decisions they are able to make
  about what ads are the ones the individual most likely will click on
  and disclose with a successful purchase. As a result this makes the
  placed advertisement more valuable for ad service and therefore more
  expensive to the advertisers, because of a high precision. Or a
  streaming provider's content recommendation is also based on heavy
  user profiling done by looking at her consumption history, tracked
  platform interactions and probably many more vectors. Another example
  is \emph{Google Traffic}
  {[}\protect\hyperlink{ref-web_2007_introducing-google-traffic}{49}{]}
  {[}\protect\hyperlink{ref-web_2016_wikipedia_google-traffic}{50}{]}, a
  service, integrated as a feature in \emph{Google Maps}, which is
  Google's web mapping service. \emph{Google Traffic} visualises
  real-time traffic conditions, when using \emph{Maps} as a navigation
  assistant, to provide the user with a selection of possible paths, but
  enriched with duration, that takes such conditions into account. The
  data, required to offer these information, is supplied by mobile
  devices, constantly sending GPS coordinates with a timestamp into
  Google's infrastructure. This, however, only is made possible, because
  Google's services are widely used in addition to the fact that the
  majority of mobile devices
  {[}\protect\hyperlink{ref-graphic_2016_global-mobile-os-market-share}{51}{]}
  is driven by Android, an mobile operating system developed by Google,
  that deeply integrates with it's services. For this case the same
  assertion can be made - the more constantly streaming geo-location
  data, the more precise the information are about traffic conditions.
  Since this information demands the real-time aspect, adding time to
  the equation, add a other dimension of complexity to problem.
\item
  while the impact on our society of this first example group might be
  doubtable, a change of perspective opens up a different range of
  application areas. Such as

  \begin{itemize}
  \tightlist
  \item
    planing and managing human resources for situations, like e.g.~big
    events or emergency situations where attendees might need some help
    {[}\protect\hyperlink{ref-estimating-the-locations-of-emergency-events-from-twitter-streams_2014}{52}{]}
  \item
    predicting infrastructure workloads {[}TODO
    http://ieeexplore.ieee.org/document/7336197/{]}
  \item
    making more accurate diagnostics to improve their therapy
    {[}\protect\hyperlink{ref-the-practice-of-predictive-analytics-in-healthcare_2013}{53}{]}
  \item
    finding patters in climate changes, which otherwise wouldn't be
    detected
    {[}\protect\hyperlink{ref-data-collection-for-climate-changes_2014}{54}{]}.
  \end{itemize}
\item
  Through all these examples, some of them might not necessarily founded
  on personal data, whereas others primarily depend on them and yet
  others only implicitly rely on data collected from individuals. As
  always, it depends on the purpose - also known as \emph{business
  model} - but it seems to be consensual, that it all comes down to
  improving and enhancing the collector's product in order to satisfy
  the customers - and that on the other hand depends on what is meant to
  be the product and who is seen as customers.
\item
  Putting a top 10 list of industries using utilizing \emph{Big Data}
  {[}\protect\hyperlink{ref-graphic_2015_applications-of-big-data-in-10-industry-verticals}{55}{]}
  right next to visualization showing categories of personal data
  targeted by data collectors\\
  {[}\protect\hyperlink{ref-graphic_2012_personal-data-ecosystem}{56}{]},
  at least 7\footnote{Banking and Securities; Communication, Media \&
    Entertainment; Healthcare Providers; Government; Insurance; Retail
    \& Wholesale Trade; Energy \& Utilities} of these industries can be
  identified as data collectors, whereas less then a half\footnote{Banking
    and Securities; Communication, Media \& Entertainment; Insurance;
    Energy \& Utilities} are taking part of being a \emph{Data Broker},
  but almost all of them are using people's personal data, whether
  collected by themselves or acquired from \emph{Data Broker}.
\item
  At this point it's save to say, that \emph{Personal Data} is either
  seen directly as a product, especially from a Dater Broker's point of
  view, or indirectly due to it's essential part in \emph{Big Data}
  practices. The former generates direct revenue by selling these data
  and the latter might affect a business's product quality in a positive
  manner and thereby increasing revenue as well.
\item
  At the end it all comes down to understanding the human being and why
  she behaves as she does. The challenge is not only to compute certain
  motives but rather concluding to the right ones. When analyzing
  computed results with the corresponding data models and trying to
  conclude, it is important to keep in mind, that correlation is by far
  no proof of causation.
\item
  individuals then get in role of selling/offering it's own data to
  those who were previously collecting them
\end{itemize}

\section{Related Work}\label{related-work}

The idea of a digital vault, controlled and maintained by it's owner,
the individual, isn't that new. Holding her most sensitive and valuable
collections of bits and bytes, protected from all these data brokers and
authorities, while interacting with the digital and physical world,
opening and closing it's door from time to time, to either put something
important for her inside or retrieving an information important for
someone else. While in the mid and late 2000s the growth of computer
performance and capacity were crossing it's zenith (see Moore's Law
{[}\protect\hyperlink{ref-paper_1965_moors-law}{57}{]}), at the same
time the internet was starting to become a key part in many people's
lives and in society as a whole. Facilitated by these circumstances,
\emph{cloud computing} has been on the rise, causing the shift towards
parallel distributed processing and patterns alike. Thereby making it
possible to rethink solutions from the past and trying to go new ways,
namely the breakthrough 2007 in \emph{neuronal networks} cutesy of G.
Hinton
{[}\protect\hyperlink{ref-podcast_2015_cre-neuronale-netze}{58}{]}. As a
result, fields like \emph{deep machine learning}, \emph{big data
analytics} and most recently \emph{data mining}, were gaining a wide
range of attention. In almost any industry a greater amount of resources
is invested in these areas
{[}\protect\hyperlink{ref-web_2016_industries-intention-to-invest-in-big-data}{59}{]}.

The initial research motivation can be seen as a counter-movement away
from the \emph{cloud}, starting to focus again on privacy, the
individual and it's digital alter ego.

From simple middleware-solutions, via full-fledged software-based
platforms, through embedded hardware devices, a great variety of
approaches were starting to appear in the mid 2000s until this day. A
side effect was, that over time various research teams and projects have
invented and coined different terms, all referring to the same concept.
The following list shows some examples \emph{(alphabetical order)}:

\begin{itemize}
\tightlist
\item
  Databox
\item
  Identity Manager
\item
  Personal \ldots{}

  \begin{itemize}
  \tightlist
  \item
    Agent
  \item
    Container
  \item
    Data Store/Service/Stream (PDS)
  \item
    Data Vault
  \item
    Information Hub
  \item
    Information Management System (PIMS)
  \end{itemize}
\item
  Vendor Relationship Management (VRM)
\end{itemize}

\subsection{Research}\label{research}

One of the first research projects is \emph{ProjectVRM}, which
originated from \emph{Berkman Center for Internet \& Society} at
\emph{Harvard University}. As it's name implies, it was inspired by the
idea of turning the concepts of a \emph{Customer Relationship
Management} (CRM) upside down. This puts the vendor's customers back in
charge of their data priorly managed by the vendors. It also solves the
problem of unintended data redundancy. Over time the project has growing
to the largest and most influential in this research field. It
transformed into an umbrella and hub for all kinds of projects and
research related to that topic
{[}\protect\hyperlink{ref-web_2016_projectvrm_development-work}{60}{]},
whether it's frameworks or standards, services offering e.g.~privacy
protection, reference implementations, applications, software or
hardware components. \emph{VRM} became more and more a synonym for a set
of principles
{[}\protect\hyperlink{ref-web_2016_projectvrm_principles}{61}{]},
including for example \emph{``Customers must have control of data they
generate and gather. {[}They{]} must be able to assert their own terms
of engagement.''} These principles can be found in various ways across a
lot of research done within this area.

Another research that is worth mentioning, because of the foundational
work it has been done, is the european funded project called
\emph{Trusted Architecture for Securely Shared Service} (TAS3). The
project led to a open source reference implementation called
\emph{ZXID}.\footnote{more information on the project, the code and the
  author, Sampo Kellomäki, can be found under \emph{zxid.org}} The major
goal was, to develop an architecture, that takes all involved parties
into account, whether it's commercial businesses (vendors) or it's users
(customers), in order to fit into more sophisticated and dynamic
processes, but at the same time demanding a high level of user-centric
security facilitate i.a. by a developed policy framework. Due to these
requirements the architecture ended up being rather complex
{[}\protect\hyperlink{ref-graphic_2011_architecture_components-of-organization-domain}{62}{]}.
\emph{ZXID} as it's implementation incorporates several standards like
SAML 2.0\footnote{Security Assertion Markup Language 2.0} and
XACML,\footnote{eXtensible Access Control Markup Language} has only
three third-party dependencies which are \emph{OpenSSL}, \emph{cURL
(libcurl)} and \emph{zlib} and as of now it supports Java, PHP and Perl.
The project lasted for a period of 4 years, but after it ended in 2011,
the research work has pursued i.a. by the \emph{Liberty Alliance
Project}, which is now part of the \emph{Kantara Initiative}
{[}\protect\hyperlink{ref-web_kantara-initiative}{63}{]}, including all
documents and results. These results were taken up occasionally,
recently from the IEEE
{[}\protect\hyperlink{ref-paper_2014_personal-data-store-approach}{64}{]}.

A research project, which is probably the closest to what this document
aims to create, bears the name \emph{openPDS}
{[}\protect\hyperlink{ref-paper_2012_openpds_on-trusted-use-of-large-scale-personal-data}{65}{]}
and is done by \emph{Humans Dynamics Lab}
{[}\protect\hyperlink{ref-web_mit_openpds-safeanswers-project-page}{66}{]},
which is part of \emph{MIT Media Laboratories}. Despite the usual
concepts of a \emph{PDS}, it introduces multi-platform components and
user interfaces including a mobile devices and separating the
persistence layer physically at the same time. This facilitates
administrative tasks regardless of the data owner's position and time.
Moreover, with their idea of \emph{SafeAnswers}
{[}\protect\hyperlink{ref-paper_2014_openpds_protecting-privacy-of-meta-data-through-safeanswers}{67}{]},
the team even goes a step further. The concept behind that, is based
around \emph{remote code execution}, briefly described in
\protect\hyperlink{header-applying-for-a-loan-and-checking-creditworthiness}{one
of the user stories during the first chapter}. It abstracts the concept
of a data request to a more human-understandable level, a simple
question. This question consists of two representation: (A) a short
explanation of what the data consumer wants to know and which data might
be involved and thus what information a data consumer actually will
receive, instead of raw data the consumer could then use for all kinds
of purposes e.g. data aggregation or mining. Aside from that, the
request payload also includes (B) a code-based representation, which
gets executed in a sandbox on the data owner's \emph{PDS} system with
the necessary data as arguments. The resulting output is answer and
response all in once.

Aside from all the research projects done within the scientific context,
applications with a commercial interest were starting to occur in a
variety of sectors, too. Microsoft's HealthVault
{[}\protect\hyperlink{ref-web_microsoft_healthvault}{68}{]}, for
example, which aims to replace all the paper-based patient file and
combine them in one digital version. This results in a patient-centered
medical data and documents archive, helping doctors to make the most
accurate decisions on medical treatment.

\emph{Meeco} {[}\protect\hyperlink{ref-web_meeco_how-it-works}{69}{]}
{[}\protect\hyperlink{ref-slides_2015_meeco-case-study}{70}{]}, based on
the MyData-Project
{[}whitepaper\_2014\_mydata-a-nordic-model-for-human-centered-personal-data-management-and-processing{]},
which essentially just cuts out the advertisement service provider as a
middle man inherits that role by itself. The platform does provide the
data owners with more control over what information they reveal, but it
doesn't go the so eagerly demanded next step, which would means real
decoupling from the advertisement market and finding a suitable business
model that focuses on the data owner, instead of surrounding them with
just another walled garden.

A recently announced project, sponsored by Germany's \emph{Federal
Ministry of Education and Research}, but developed and maintained
primarily by \emph{Fraunhofer-Gesellschaft} in cooperation with several
private companies like \emph{PricewaterhouseCoopers AG},
\emph{Volkswagen AG}, \emph{thyssenkrupp AG} or \emph{REWE Systems
GmbH}, is the so called \emph{Industrial Data Space}
{[}\protect\hyperlink{ref-web_industrial-data-space}{71}{]}. The project
unifies both, research and commercial interests and runs over time
period of three years until the third quarter of 2018. It aims to
\emph{``{[}\ldots{}{]} to facilitate the secure exchange and easy
linkage of data in business ecosystems''}, where at the same time
\emph{``{[}\ldots{}{]} ensuring digital sovereignty of data owners''}
{[}\protect\hyperlink{ref-whitepaper_2016_industrial-data-space}{72}{]}.
It will be interesting to see how these two, yet rather distinct
objectives, will come together in the future. Based on the white paper,
the project's focus mainly seems to lie in enabling and standardizing
the way companies collect, exchange and aggregate data with each other
across process chains to ensure high interoperability and accessibility.

Hereafter a selective list can be found of further research projects,
work and commercial products regarding the issue around \emph{personal
data}:

\subsubsection{Research}\label{research-1}

\begin{itemize}
\tightlist
\item
  Higgins {[}https://www.eclipse.org/higgins/{]}
\item
  Hub-of-All-Things {[}http://hubofallthings.com/what-is-the-hat/{]}
\item
  ownyourinfo {[}http://www.ownyourinfo.com{]}
\item
  PAGORA {[}http://www.paoga.com{]}
\item
  PRIME/PrimeLife {[}https://www.prime-project.eu,
  http://primelife.ercim.eu/{]}
\item
  databox.me (reference implementation of the
  \emph{\href{https://github.com/solid/solid}{Solid framework}})
\item
  Polis (greek research project from 2008)
  {[}http://polis.ee.duth.gr/Polis/index.php{]}
\end{itemize}

\subsubsection{Organisations}\label{organisations}

\begin{itemize}
\tightlist
\item
  Open Identity Exchange
  {[}http://openidentityexchange.org/resources/white-papers/{]}
\item
  Qiy Foundation {[}https://www.qiyfoundation.org/{]}
\end{itemize}

\subsubsection{Commercial Products}\label{commercial-products}

\begin{itemize}
\tightlist
\item
  MyData {[}https://mydatafi.wordpress.com/{]}
\item
  RESPECT network {[}https://www.respectnetwork.com/{]}
\item
  aWise AEGIS {[}http://www.ewise.com/aegis{]}
\end{itemize}

\section{Standards and
Specifications}\label{standards-and-specifications}

When developing an \emph{Open Specification} it comes naturally to build
upon open technologies, which shall be understood as open standards and
open source; \emph{open} in the sense of \emph{unrestricted accessible
by everybody}; not to be confused with free - as in \emph{freedom} -
software. In this case, advocating such a an openness enables not only
to develop implementations of the specification in a collaborative way,
but also the specification itself, and makes it possible for anyone who
is interested to participate or even to contribute. For everyone who
just want to use open technologies, a license defining rules and
conditions is typically enclosed somehow. But regardless of the
motivation everybody who is interested in getting to know how these
hard- or software blackboxes-by-design are actually working, is thus
able to look into it.

So the overall attempt is to involve as much standards as possible,
because it increases the chances of interoperability and thereby it
lowers the effort, that might be needed, in order to integrate with
third parties or other APIs. Hereinafter, some of these possible
technologies will be touched on just briefly, why they might be a
reasonable choice and what purposes they might going to service.

\textbf{HTTP(S)} {[}\protect\hyperlink{ref-web_spec_http1}{73}{]}, well
known as the transport layer for the \emph{World Wide Web} is most
likely going to fulfill the same purpose in the context of this work.
Whether internal components (local or as part of a distributed system)
talk to each other or data consumers request information. Features
introduces with Version 2
{[}\protect\hyperlink{ref-web_spec_http2}{74}{]} of the protocol are yet
to be known of their relevance and corresponding use cases. The
\emph{Transport Layer Security}
{[}\protect\hyperlink{ref-web_spec_tls}{75}{]} embedded in the protocol
provides encryption during transfer, which reduces the vulnerability to
\emph{man-in-the-middle} attacks and thus ensures data integrity. Due to
it's asymmetrical cryptographic concepts used to establish a connection,
\emph{TLS} also allows to verify the integrity of the entity on the the
connection's counterside, and, depending on the integration, it could
even used for authentication. \emph{Websockes}
{[}\protect\hyperlink{ref-web_spec_websockets}{76}{]} might also be a
possibility to communicate between components or even with external
parties, which has the advantage of high efficient ongoing connections
using for real-time data exchange or remotely pending process responses,
while at the same time avoiding HTTP's long-polling abilities.

\textbf{JSON}\footnote{The JavaScript Object Notation (JSON) Data
  Interchange Format; ECMA Standard\\
  {[}\protect\hyperlink{ref-web_spec_json}{77}{]} and Internet
  Engineering Task Force RFC 7159
  {[}\protect\hyperlink{ref-web_rfc_json}{78}{]}} is an alternative data
serialization format to XML, heavily used in web contexts to transfer
data via \emph{HTTP}, whose syntax is inspired by the JavaScript
object-literal notation.

The open standard \textbf{OAuth} defines a process flow for authorizing
third parties to access externally hosted resources, such as the user's
profile image from \emph{facebook}. The authorisation validation is done
with the help of a previously generated token. However generating and
supplying such a token can be initiated in a variety of ways depending
on the situation, e.g.~with the user entering her credentials
(\texttt{grant\_type=authorization\_code}). This design mistakenly
{[}\protect\hyperlink{ref-web_2012_problem-with-oauth-for-authentication}{79}{]}
lead to \emph{OAuth} integrations with the intention to provide an
authentication service whether as an alternative or as an addition to
existing in-house solution. Therewith the application authors pass the
responsibility on to the OAuth-supporting data providers. While
\emph{version 1.0a} {[}\protect\hyperlink{ref-web_spec_oauth-1a}{80}{]},
seen as a protocol, provides integrity for transferred data by using
signatures and confidentiality by encrypting data ahead of transfer.
Whereas \emph{version 2.0}
{[}\protect\hyperlink{ref-web_spec_oauth-2}{81}{]}, labeled as a
framework, just requires \emph{TLS}. It also includes certain process
flows for specific platforms, such as \emph{``web applications, desktop
applications, mobile phones, and living room devices''}
{[}\protect\hyperlink{ref-web_2016_oauth-2}{82}{]}.

With \textbf{OpenID} on the other side, the authenticity of a requesting
user gets verified, which is by design. An in-depth description of the
whole process can be found in the protocol's same-titled open standard.
With decentralisation kept in mind, the protocols's nature encourages to
design a distributed application architecture, similar to the idea
behind \emph{microservices}, but without owning all services involved,
\emph{decentralized authentication as a service} so to speak. An
application owner doesn't have to write or implement it's own user
management system, instead it is sufficient to just integrate these
parts from the standard need to support signing in with \emph{OpenID}.
Equally the user is not required to register a new account whenever it
is necessary, instead she can use her \emph{OpenID}, already created by
another identity provider, to authenticate with the application. The
extension \emph{OpenID Attribute Exchange} allows to import additional
profile data. \emph{OpenID Connect}
{[}\protect\hyperlink{ref-web_spec_openid-connect-1}{83}{]} is the third
iteration of the OpenID technology \emph{Connect} is to OpenID what
\emph{facebook connect} is to \emph{facebook}, except for the additional
authentication layer, which is build upon \emph{OAuth2.0} and therefore
enables, aside from authorisation mechanisms, third parties to
authenticate an OpenID-user and makes certain data available about that
account via REST interface.

If it's necessary for certain components, as part of a distributed
software, to make them stateless, apart from changing the architecture
so that the state at that point is not needed anymore, the only other
option would be to carry the state along (TODO: or ``passing the state
around''). This is a common use case for a \textbf{JSON Web Token}
\emph{(JWT)} {[}\protect\hyperlink{ref-web_spec_json-web-token}{84}{]}.
A \emph{JWT}, as it's name implies, is syntactically speaking formatted
as \emph{JSON}, but URI-safe into \emph{Base64} encoded, before it gets
transferred. The token itself holds the state. Here is where the use of
\emph{HTTP} comes in handy, because the token can be stored within the
HTTP header and therefore can be passed through all communication
points, where then certain data could be readout and therewith get
verified. Such a token typically consists of three parts: information
about itself, a payload, which can be arbitrary data such as user or
state information, and a signature; all separated with a period.
Additional standards define encryption \emph{(JWE\footnote{JSON Web
  Encryption, Internet Engineering Task Force RFC 7516
  {[}\protect\hyperlink{ref-web_spec_json-web-encryption}{85}{]}})} to
ensure confidentiality and signatures \emph{(JWS\footnote{JSON Web
  Signature, Internet Engineering Task Force RFC 7515
  {[}\protect\hyperlink{ref-web_spec_json-web-signature}{86}{]}})} to
preserve integrity of it's contents. Using a \emph{JWT} for
authentication purposes, that is described as \emph{stateless
authentication}, because the verifying entity doesn't need to be aware
of session IDs or know anything about a certain state. So instead of the
backend interface being constrained to check a state
(\texttt{isLoggedIn(sessionId)} or \texttt{isAuthorized(sessionId)}) on
every incoming request in order to verify permissions, it just needs

\textbf{REST(ful)}\footnote{\emph{Representational State Transfer},
  introduces by Roy Fielding in his doctoral dissertation
  {[}\protect\hyperlink{ref-web_spec_rest}{87}{]}} is a common set of
principles to design web resources communication, primarily
server-client relations, in a more generic and thereby interoperable
way. Aside from hierarchically structured URIs, which reflect semantic
meanings, it involves a group of rudimentary vocabulary\footnote{knows
  as HTTP Methods or Verbs
  {[}\protect\hyperlink{ref-web_spec_http-methods}{88}{]} (e.g.~GET,
  OPTIONS, PUT, DELETE)} to provide basic Create-Read-Update-Delete
operations across distributed systems. The entire request need to
contain everything that is required to get proceeded, e.g.~state data
and possibly authentication. These operation normally wont get applied
directly to the responsible component. Instead the whole system (or
certain services) exposes a restful API, with which a third party can
then interact.

The \emph{QL} in \textbf{GraphQL}
{[}\protect\hyperlink{ref-web_spec_graphql}{89}{]} stands for
\emph{query language}. It's goal is to abstract multiple data sources in
order to unify them under one API and make all containing data
queryable, including all relating data points. The returned data,
emitted in JSON syntax, can exhibit graph-like structures, meaning
multiple data points, that might be somehow related to each other, or in
other words: indirectly ``linked'' through each other. These, naturally
deep-leveled structures, can be described by the syntax of the query
language.

The term \textbf{Semantic Web} bundles a conglomerate of standards
addressing syntax, schemas, assess control and integration around the
idea of \emph{web of data} to \emph{``allow data being shared and reused
across''} {[}web\_2016\_w3c\_semantic-web-activity{]} or within several
scopes and contexts. Alongside several others, the following three
standards have a certain relevance to that concept. RDF\footnote{Resource
  Description Framework {[}\protect\hyperlink{ref-web_w3c-tr_rdf}{90}{]}}
basically defines the syntax. OWL\footnote{Web Ontology Language
  {[}\protect\hyperlink{ref-web_w3c-tr_owl}{91}{]}} provides the
guidelines on how the semantics and schemas should be defined and with
SPARQL {[}\protect\hyperlink{ref-web_w3c-tr_sparql}{92}{]}, the query
language for the RDF format, the data can be retrieved. A picture
emerges in which the web is used as a database, queried by URIs with a
query language. An example would be a person's email address, which is
available under a specific domain (preferable owned by that person) - or
to be more precise, an URI \emph{(WebID)
{[}\protect\hyperlink{ref-web_w3c-draft_webid}{93}{]}} - and provided in
a certain syntax \emph{(RDF)} and tagged with the semantic \emph{(OWL)}
of a email address; all embedded in a valid html page. This information
can be queried \emph{(SPARQL)}, which requires at least the URI, working
as a unique identifier. While defining the standards, an importancy was
to define a syntax which is also valid markup, in order to maintain a
single source of trough and save redundant work. Related to this topic
is the work on a specification called \textbf{Solid}.\footnote{social
  linked data {[}\protect\hyperlink{ref-web_spec_solid}{94}{]}} Based on
the \emph{Linked Data} principals, that are facilitated through the
standards just mentioned and the \emph{WebAccessControl}
{[}\protect\hyperlink{ref-web_2016_wiki_webaccesscontrol}{95}{]} system,
the project focuses on decentralization and personal data. A reference
implementation called \emph{databox}
{[}\protect\hyperlink{ref-web_2016_demo_databox}{96}{]} combines all
these technologies and is build on top.

The concept of application (or software) \textbf{container} is about
encapsulating runtime environments by introducing an additional layer of
abstraction. A container bundles just the software dependencies
(e.g.~binaries) that are absolutely necessary so that the enclosed
program is able to run properly. The actual container separation is
done, aside from others, with the help of two features provided by the
Linux kernel. \emph{Cgroups},\footnote{control groups
  {[}\protect\hyperlink{ref-web_2015_cgroup-doc}{97}{]}} which define or
restrict how much of the existing resources a group of processes
(e.g.~CPU, memory or network) can use. Whereas \emph{namespaces}
{[}\protect\hyperlink{ref-web_2016_kernel-namespace}{98}{]} define or
restrict what parts of the system can be accessed or seen by a process
(e.g.~filesystem, user, other processes). The idea of encapsulating
programs from the operating system-level is not new, Technologies, such
as \emph{libvirt}, \emph{systemd-nspawn}, \emph{jails}, or
\emph{hypervisors} (e.g.~VMware, KVM, virtualbox) have been used for
years, but were usually too cumbersome and never reached a great level
of convenience, so that only people with a certain expertise were able
to handle systems build upon virtualization, but people with other
backgrounds couldn't and weren't that much interested. Until
\emph{Docker} and \emph{rkt} emerged. After some years of separated
work, both authors, and others, recently joined forces in the \emph{Open
Container Initiative}
{[}\protect\hyperlink{ref-web_2016_open-container-initiative}{99}{]},
which aims to harmonize the diverged landscape and start building common
ground to ensure a higher interoperability, and that in turn is
requisite for orchestration. It also marks the initial draft of the
specifications for runtime
{[}\protect\hyperlink{ref-web_oci-spec_runtime}{100}{]} and image
{[}\protect\hyperlink{ref-web_oci-spec_image}{101}{]} definition, on
which the work is still ongoing. This concept of \emph{containerization}
also inherits the a ability known from \emph{emulation}, because it
allows a certain set of software to run on a system that otherwise is
not supported, e.g.~mobile devices. It only requires the runtime to be
working.

\chapter{Core Principles}\label{core-principles}

In the following chapter the certain core principles of the system will
be examined

NOTE: here we discuss a variety of possibilities --\textgreater{}
conceptual work

\section{Data Ownership}\label{data-ownership}

\begin{itemize}
\tightlist
\item
  user-centric, full control
\end{itemize}

\section{Identity Verification}\label{identity-verification}

\begin{itemize}
\tightlist
\item
  maybe go with a Signing/verifying Authority (aka CA)

  \begin{itemize}
  \tightlist
  \item
    do I trust the gov or certain companies more? Which interests do
    these Role/Stakeholder have?
  \item
    revoking the cert which provides the authenticity of the
    individual's digital identity should only be possible with a
    two-factor secret. One part of this secret is owned by the CA and
    the other half has the individual behind the personal API
  \end{itemize}
\item
  TODO: look into

  \begin{itemize}
  \tightlist
  \item
    PKI
  \item
    ePerso
  \item
    E-Post/de-mail
  \end{itemize}
\item
  Authentication
\end{itemize}

\section{Authorisation (remove in favour of data
access?)}\label{authorisation-remove-in-favour-of-data-access}

\begin{itemize}
\tightlist
\item
  NOTE: does not mean this tool authenticates it's owner against third
  party platforms like OpenID does. but it could play the role of the 2n
  factor in a multi-factor authentication process (if the
  mobile-device-architecture was chosen)
\item
  refers primarily to the process of a data consumer (third party, which
  needs the data for whatever reason) verifies her admission to request
\end{itemize}

\section{Authentic Data}\label{authentic-data}

\begin{itemize}
\tightlist
\item
  is this data (in this case identity) certified or not (results in
  higher value)
\end{itemize}

\section{Supervised Data Access}\label{supervised-data-access}

\begin{itemize}
\tightlist
\item
  pure/plain data request/resonse
\item
  remote computation/execution (assuming there is no client for the
  consumer) like https://webtask.io/
\end{itemize}

\section{Encapsulation}\label{encapsulation}

\begin{itemize}
\tightlist
\item
  containerization (coreos, rkt, mirageos aka unikernal)
\end{itemize}

\section{Open Development}\label{open-development}

\begin{itemize}
\item
  which and why open standards
\item
  why open source
\item
  collaborative transparent development
\item
  Hosting \& Administration

  \begin{itemize}
  \tightlist
  \item
    DYI
  \item
    Usability
  \end{itemize}
\end{itemize}

\chapter{Requirements}\label{requirements}

The subsequent requirements shall be served as a list of features on the
on hand, to get an idea about what the end result might look like, and
to give an overview about priorities (can/could, may/might, should,
must/have to) on the other hand. Following chapters may contain specific
references to the requirements listed below.

\section{The System}\label{the-system}

The system perspective includes the overall architecture, fundamental
building blocks, such as storage \& persistence layer, business logic
and outmost interfaces.

\subsubsection{Architecture:}\label{architecture}

\textbf{\emph{\protect\hypertarget{sa01}{}{S.A.01}} - Portability}\\
+ all components should be relocatable and function independently

\begin{itemize}
\tightlist
\item
  the different components from the system should be able to get
  relocated and remain to be fully functional
\end{itemize}

\subsubsection{Persistence:}\label{persistence}

\emph{\protect\hypertarget{sp01}{}{S.P.01}}: data may only leave the
system if it's absolutely necessary and no other option exists to
preserve the goal of that process

\emph{\protect\hypertarget{sp02}{}{S.P.02}}: the data structure and data
models must show high flexibility and must not consist of strong
relations

\emph{\protect\hypertarget{sp03}{}{S.P.03}}: the storage (engine) should
be supported on different platforms (incl. mobile devices)

\subsubsection{Interfaces:}\label{interfaces}

\textbf{\emph{\protect\hypertarget{si01}{}{S.I.01}} - Documentation}\\
All interfaces between components have to be documented, in a way that
the components themselves can be replaced without any impact to the rest
of the system

\section{The User}\label{the-user}

Not only humans that are using the resulting software are seen as a
\emph{user}. Also developers and other contributors may use the
software, documentations or program code from a technical standpoint.

\subsubsection{Administration:}\label{administration}

\emph{\protect\hypertarget{ua01}{}{U.A.01}}:

\subsubsection{Management \&
Organisation:}\label{management-organisation}

\emph{\protect\hypertarget{umo01}{}{U.MO.01}}:

\section{The Software}\label{the-software}

\subsubsection{Visual User Interface:}\label{visual-user-interface}

\textbf{\emph{\protect\hypertarget{pviu01}{}{P.VIU.01}} - Responsive
user interface}\\
The visual user interface has to be responsive to the available space

\textbf{\emph{\protect\hypertarget{pviu02}{}{P.VIU.02}} - Platform
support}\\
+ must at least web UI (provided by the server)

\begin{itemize}
\tightlist
\item
  should mobile to benefit from notifications and storing data on that
  device
\end{itemize}

\textbf{\emph{\protect\hypertarget{pviu03}{}{P.VIU.03}} - Access
Profiles}\\
The user should be capable of filtering, sorting and searching through
that list of profiles

\textbf{\emph{\protect\hypertarget{pviu04}{}{P.VIU.04}} - Access
History}~ The user must be provided with a list of all past permission
requests and data accesses. This tool should have filter, search and
sort mechanisms.

\begin{itemize}
\tightlist
\item
  requires \protect\hyperlink{pb01}{access logging}
\end{itemize}

\subsubsection{Interactions:}\label{interactions}

\textbf{\emph{\protect\hypertarget{pi01}{}{P.I.01}} - Effort}\\
Common interactions processes, like changing \emph{profile data},
importing data sets or manage \emph{permission request} have to be
require as little effort as possible. This means short UI response time
on the one hand and and as less single inputs as possible to complete a
task.

\textbf{\emph{\protect\hypertarget{pi02}{}{P.I.02}} - Design}\\
The visual user interface must be designed and structured in such a way
that is is highly intuitive for the user to operate. Thus, it is
important e.g.~to use meaningful icons and appropriate labels. TODO:
emphasize more UI aspects (or not)

\textbf{\emph{\protect\hypertarget{pi03}{}{P.I.03}} - Notifications}\\
The user should be notified about every interaction with the
\emph{PDaaS} originated by a third party immediately after it's
occurrence, but she must get notified at least about every
\emph{permission request}. This behaviour should be configurable;
depending on the \emph{permission type} and on every \emph{access
profile}. Regardless of the configuration the notifications themselves
must show up and pending user interactions must be indicated in the user
interface.

\textbf{\emph{\protect\hypertarget{pi04}{}{P.I.04}} - Permission Request
Review}\\

\subsubsection{Behaviour:}\label{behaviour}

\textbf{\emph{\protect\hypertarget{pb01}{}{P.B.01}} - Access Logging}\\
All interactions and changes in the persistence layer should be logged.
At least all data request must be logged. Such log is the foundation of
the \emph{access history}, with this the user is able to keep track of
and look up past accesses.

\textbf{\emph{\protect\hypertarget{pb02}{}{P.B.02}} - Real time}\\
+ web ui in browser on desktop should be connected to the server through
websockets to support real time (e.g.~permission request got reviewed on
mobile device, but notification indicator reflects ``still pending'')

\begin{itemize}
\tightlist
\item
  if just one client is associated to the system, real time (in the
  sense of keeping UI state up to date) would not be required
\end{itemize}

\section{The Functionality}\label{the-functionality}

\emph{\protect\hypertarget{f01}{}{F.01}}: The system

the process of granting access to certain data points has to be
effortless and intuitive

data owners

data consumers

data owners can use permission templates to predefine rules for later
application to data consumer requests

a process involving data transaction must always be initiated by the
data owner

the data owner should be able to (semi-)automate additional an ongoing
data imports from multiple data sources (e.g.~IoT, browser plugin)

\chapter{Design Discussion}\label{design-discussion}

The following chapter documents the processes of some design decision
makings, examines possible issues emerging alongside and discuses
different solutions obtained from several perspectives in order to
evaluate their advantages and disadvantages. Probably not every issue
will get it's\\
deserved rome , but major aspects will be addressed.

\section{Architecture}\label{architecture-1}

\begin{itemize}
\tightlist
\item
  showing possible directions, e.g.:

  \begin{itemize}
  \tightlist
  \item
    cloud or local storage
  \item
    which components can go where
  \item
    remote execution, to prevent data from leaving the system
  \end{itemize}
\end{itemize}

\subsection{Overview}\label{overview}

\begin{itemize}
\tightlist
\item
  distributed architecture (e.g.~notification/queue server + mobile
  device for persistence and administration)
\end{itemize}

\subsection{Components}\label{components}

\subsection{Plugins}\label{plugins}

\begin{itemize}
\tightlist
\item
  but for what? and not harm security at the same time? maybe just for
  data types and admin UI to display analytical (time based) data in
  other ways
\item
  what about extensions (see iOS 10) to let other apps consume data;
  only on a mobile device and only if the data is stored there
\end{itemize}

\section{Data}\label{data}

\begin{itemize}
\item
  keep in mind to make it all somehow extendible, e.g.~by using and
  storing corresponding schemas
\item
  NOTE: step numbers marked with a \texttt{*} are somehow tasks which
  are happening in the background and don't require any user interaction
\end{itemize}

\subsection{Modelling}\label{modelling}

\subsection{Categories (or Classes)}\label{categories-or-classes}

\subsection{Types}\label{types}

\subsection{Persistence}\label{persistence-1}

\begin{itemize}
\tightlist
\item
  database requirements
\end{itemize}

\subsection{Access \& Permission}\label{access-permission}

\begin{itemize}
\tightlist
\item
  data needs to have an expiration date
\end{itemize}

\textbf{IF01 - Authorizing a consumer to request certain data}

\begin{enumerate}
\def\labelenumi{\arabic{enumi})}
\tightlist
\item
  owner creates a new endpoint URI (like
  \emph{pdaas.ownersdomain.tld/e/consumer-name}) within the
  \emph{management user interface}
\item
  owner passes this URI on to the \emph{consumer}, e.g.~through
  submitting a form or using any arbitrary, eventually insecure channel
  3*) consumer need to call this URI for the fist time to verify it's
  authenticity
\item
  owner then get's a notification which asks her for permissions to
  access certain data under the listed conditions 5*) consumer will be
  informed about the outcome of the owner's decision (NOTE: alongside
  with some details? how do they look like? XXX need to be in the spec)
\end{enumerate}

\subsection{Consumption (data inflow)}\label{consumption-data-inflow}

\begin{itemize}
\tightlist
\item
  how data will get into the system
\item
  hwo is the user able to do that, and how does it works
\end{itemize}

\subsubsection{Manually}\label{manually}

\subsubsection{Automatically}\label{automatically}

\subsection{Emission (data outflow)}\label{emission-data-outflow}

\begin{itemize}
\item
  depending on what category of data, they might need to get anonymized
  somehow before they leave the system
\item
  oAuth (1.0a and 2) requires consumers to register upfront. Since the
  current flow indicates that the initial step is done by the owner,
  that would cause an overhead in user interactions. Although the owner
  already \emph{authorized} the consumer simply by submitting a unique
  URI
  (\texttt{pdaas-server.tld/register?crt=CONSUMER\_REGISTER\_TOKEN}), of
  which the \texttt{crt} is considered private. Even though the
  registration provides the consumer with mandatory information such as
  a consumer identifier (v1: \texttt{oauth\_consumer\_key}, v2:
  \texttt{client\_id}) and, depending on the client type, a secret (see
  https://tools.ietf.org/html/rfc6749\#section-2), this process it is
  not part the specification
  (https://oauth.net/core/1.0a/\#rfc.section.4.2,
  https://tools.ietf.org/html/rfc6749\#section-2). This enables the
  possibility of integrating oAuth into the consumer registration flow
  by using the \texttt{CONSUMER\_REGISTRATION\_TOKEN} as oAuth's
  \emph{client identifier}. The lack of credentials (v1:
  \texttt{auth\_consumer\_secret}, v2: \texttt{client\_secret}) would
  require transferring the consumer identifier done over a secure
  channel (e.g.~TLS). That would leave \emph{oAuth2} as the version of
  choice, since it relies on \emph{HTTPS} adn therefore makes the
  \emph{secret} optional. Where on the other side oAuth 1.0a requires a
  \emph{secret} to create a signature in order to support insecure
  connections..
\item
  A general and URI for 3rd parties to register (aka requesting
  authentication) would raise the issue of dealing with spam request and
  how to distinct these from the actual ones.
\end{itemize}

\subsection{History}\label{history}

\begin{itemize}
\tightlist
\item
  data versioning
\item
  access logs
\end{itemize}

\section{Interfaces}\label{interfaces-1}

\subsection{Internal}\label{internal}

\begin{itemize}
\tightlist
\item
  UI for Management \& Administration
\end{itemize}

\subsection{External}\label{external}

\begin{itemize}
\item
  should there be a way to somehow request information about what data
  is available/queryable, or would this be result in spam/crawler and
  security issues (also a question for the topic of
  permissions/sensibility level of certain data)
\item
  certain types of requests, depending on expire date:

  \begin{itemize}
  \tightlist
  \item
    ``ask me any time''
  \item
    ``allowed until further notice''
  \item
    one-time permission (but respecting certain http error codes and
    possible timeout - that might not count)
  \end{itemize}
\end{itemize}

\subsection{Authentication}\label{authentication}

\begin{itemize}
\tightlist
\item
  regarding oAuth as authentication: Priorly users tented to reuse their
  password for different account, nowadays they but alsod tent to get
  tired of creating new accounts and profiles over and over again
  instead of having just one account for everything
  {[}\protect\hyperlink{ref-web_2009-success-of-facebook-connect}{102}{]}.
\end{itemize}

Thus the platform owners leave the responsibility of

Many websites and platforms understand those \emph{login-with
\$platformName} mechanisms as an outsourced service that handles all
security- and user-related tasks.

what about token stealing when using jwt?

\chapter{Specification}\label{specification}

\begin{itemize}
\tightlist
\item
  what does \emph{open} in Open Specification even mean?
\end{itemize}

\section{\texorpdfstring{Processes (TODO: find another word; ``Protocol
flows''?)}{Processes (TODO: find another word; Protocol flows?)}}\label{processes-todo-find-another-word-protocol-flows}

\section{Application Programming
Interfaces}\label{application-programming-interfaces}

\section{Graphical User Interfaces}\label{graphical-user-interfaces}

\section{Security}\label{security}

\begin{itemize}
\item
  the downside of having not just parts of the personal data in
  different places (which is currently the common way to store), is in
  case of security breach, it would increase the possible damage by an
  exponential rate Thereby all data is exposed at once, instead of not
  just the parts which a single service has stored
\item
  does it matter from what origin the data request was made? how to
  check that? is the requester's server domain in the http header?
  eventually there is no way to check that, so me might need to go with
  request logging and trying to detect abnormal behaviour
\item
  is the consumer able to call the access request URI repeatedly and any
  time? (meaning will this be stateless or stateful?)
\item
  initial consumer registration would be done on a common and valid
  https:443 CA-certified connection. after transferring their cert to
  them as a response, all subsequent calls need to go to their own
  endpoint, defined as subdomains like
  \texttt{consumer-name.owners-notification-server.tld}
\end{itemize}

\subsection{Environment}\label{environment}

\subsection{Transport}\label{transport}

\begin{itemize}
\tightlist
\item
  communication between internal components \emph{must} be done in https
  only, but which ciphers? eventually even http/2?
\end{itemize}

\subsection{Storage}\label{storage}

\begin{itemize}
\tightlist
\item
  documents based DB instead of Relational DBS, because of
  structure/model flexibility
\end{itemize}

\subsection{Authentication}\label{authentication-1}

\begin{itemize}
\tightlist
\item
  how should consumer authenticate?
\end{itemize}

\section{Recommendations}\label{recommendations}

\subsection{Software Dependencies}\label{software-dependencies}

\subsection{Host Environment}\label{host-environment}

\chapter{Conclusion}\label{conclusion}

\section{\texorpdfstring{Ethical \& Social Impact (TODO: or
``Relevance'')}{Ethical \& Social Impact (TODO: or Relevance)}}\label{ethical-social-impact-todo-or-relevance}

\section{Business Models \&
Monetisation}\label{business-models-monetisation}

\begin{itemize}
\tightlist
\item
  possible resulting direct or indirect business models
\item
  owner might want to sell her data, only under her conditions.
  therefore some kind of infrastructure and process is required (such as
  payment transfer, data anonymization, market place to offer data)
\end{itemize}

\section{Challenges}\label{challenges}

\begin{itemize}
\tightlist
\item
  adoption rate of such technology
\end{itemize}

\section{Solutions}\label{solutions}

\section{Attack Scenarios}\label{attack-scenarios}

\begin{itemize}
\tightlist
\item
  single point of failure (data-wise),

  \begin{itemize}
  \tightlist
  \item
    but considering what data users already put into their social
    networks (or: thE social network: fb), they/it has already become a
    de facto data silo and is thus a single point of failure. If that
    service breaks or get down, the data from all users might be lost or
    worse (stolen). The aspect of data decentralisation achieved by
    individual data stores can be valued as positive.
  \end{itemize}
\end{itemize}

\section{Future Work}\label{future-work}

\begin{itemize}
\tightlist
\item
  maybe enable the tool to play the role of an own OpenID provider?
\item
  going one step further and train machine (predictor) by our self with
  our own data
  (https://www.technologyreview.com/s/514356/stephen-wolfram-on-personal-analytics/)
\end{itemize}

\section{Summary}\label{summary}

\begin{itemize}
\tightlist
\item
  main focus
\item
  unique features
\item
  technology stack \& standards
\item
  resources
\item
  the tool might be not a bulletproof vest, but
\end{itemize}

\hypertarget{refs}{}
\hypertarget{ref-web_2016_privacy-international-about-big-data}{}
{[}1{]} ``Big data privacy international.'' {[}Online{]}. Available:
\url{https://www.privacyinternational.org/node/8}. {[}Accessed:
15-Nov-2016{]}

\hypertarget{ref-paper_2008_discrimination-aware-data-mining}{}
{[}2{]} D. Pedreshi, S. Ruggieri, and F. Turini, ``Discrimination-aware
data mining,'' in \emph{Proceedings of the 14th ACM SIGKDD international
conference on Knowledge discovery and data mining}, 2008, pp. 560--568
{[}Online{]}. Available:
\url{http://dl.acm.org/citation.cfm?id=1401959}. {[}Accessed:
03-Nov-2016{]}

\hypertarget{ref-book_2015_ethical-it-innovation_ethical-uses-of-information-and-knowledge}{}
{[}3{]} S. Spiekermann, \emph{Ethical IT Innovation: A Value-Based
System Design Approach}. CRC Press; Taylor \& Francis Group, LLC, 2015,
pp. 66--72 {[}Online{]}. Available:
\url{https://www.crcpress.com/Ethical-IT-Innovation-A-Value-Based-System-Design-Approach/Spiekermann/p/book/9781482226355}

\hypertarget{ref-paper_1996_bias-in-computer-systems}{}
{[}4{]} B. Friedman and H. Nissenbaum, ``Bias in computer systems,''
\emph{ACM Transactions on Information Systems (TOIS)}, vol. 14, no. 3,
pp. 330--347, 1996 {[}Online{]}. Available:
\url{http://dl.acm.org/citation.cfm?id=230561}. {[}Accessed:
07-Nov-2016{]}

\hypertarget{ref-wikipedia_2016_cognitive-bias}{}
{[}5{]} ``Cognitive bias,'' \emph{Wikipedia}, Oct-2016. {[}Online{]}.
Available:
\url{https://en.wikipedia.org/w/index.php?title=Cognitive_bias\&oldid=742803386}.
{[}Accessed: 08-Nov-2016{]}

\hypertarget{ref-web_2016_big-data-is-people}{}
{[}6{]} R. Lemov, ``Why big data is actually small, personal and very
human. Aeon essays,'' 16-Jun-2016. {[}Online{]}. Available:
\url{https://aeon.co/essays/why-big-data-is-actually-small-personal-and-very-human}.
{[}Accessed: 17-Nov-2016{]}

\hypertarget{ref-video_2015_big-data-and-deep-learning_discrimination}{}
{[}7{]} A. Dewes, ``C3TV - Say hi to your new boss: How algorithms might
soon control our lives.'' 29-Dec-2015. {[}Online{]}. Available:
\url{https://media.ccc.de/v/32c3-7482-say_hi_to_your_new_boss_how_algorithms_might_soon_control_our_lives\#video\&t=1538}.
{[}Accessed: 03-Nov-2016{]}

\hypertarget{ref-web_2010_projectvrm_about}{}
{[}8{]} ``ProjectVRM - about. ProjectVRM,'' 25-Feb-2010. {[}Online{]}.
Available: \url{https://blogs.harvard.edu/vrm/about/}. {[}Accessed:
09-Nov-2016{]}

\hypertarget{ref-paper_2013_the-personal-data-store-approach-to-personal-data-security_2013}{}
{[}9{]} Tom Kirkham, Sandra Winfield, Serge Ravet, and S. Kellomaki,
``The personal data store approach to personal data security,''
\emph{IEEE Security \& Privacy}, vol. 11, no. 5, pp. 12--19, 2013.

\hypertarget{ref-whitepaper_2014_mydata-a-nordic-model-for-human-centered-personal-data-management-and-processing}{}
{[}10{]} A. Poikola, K. Kuikkaniemi, and H. Honko, ``MyData -- a nordic
model for human-centered personal data management and processing,'' pp.
1--12, Jun. 2015 {[}Online{]}. Available:
\url{https://www.lvm.fi/documents/20181/859937/MyData-nordic-model/2e9b4eb0-68d7-463b-9460-821493449a63}.
{[}Accessed: 10-Nov-2016{]}

\hypertarget{ref-web_2016_meeco-how-it-works}{}
{[}11{]} ``Meeco how it works.'' {[}Online{]}. Available:
\url{https://meeco.me/how-it-works.html}. {[}Accessed: 09-Nov-2016{]}

\hypertarget{ref-repo_2016_pdaas-spec}{}
{[}12{]} ``Open specification of the concept called personal data as a
service (pdaas). GitHub.'' {[}Online{]}. Available:
\url{https://github.com/lucendio/pdaas_spec}. {[}Accessed:
11-Nov-2016{]}

\hypertarget{ref-web_2010_projectvrm-wiki_about-vrm}{}
{[}13{]} ``ProjectVRM wiki - about VRM.'' {[}Online{]}. Available:
\url{https://cyber.harvard.edu/projectvrm/Main_Page\#About_VRM}.
{[}Accessed: 11-Nov-2016{]}

\hypertarget{ref-web_2010_projectvrm-wiki_pims-example-list}{}
{[}14{]} ``ProjectVRM wiki - list of personal information management
systems.'' {[}Online{]}. Available:
\url{https://cyber.harvard.edu/projectvrm/VRM_Development_Work\#Personal_Information_Management_Systems_.28PIMS.29}.
{[}Accessed: 11-Nov-2016{]}

\hypertarget{ref-report_2014_data-brokers}{}
{[}15{]} F. T. C. USA, ``Data brokers,'' May 2014 {[}Online{]}.
Available:
\url{https://www.ftc.gov/system/files/documents/reports/data-brokers-call-transparency-accountability-report-federal-trade-commission-may-2014/140527databrokerreport.pdf}.
{[}Accessed: 17-Nov-2016{]}

\hypertarget{ref-whitepaper_2012_the-value-of-our-digital-identity_definition}{}
{[}16{]} J. Rose, O. Rehse, and B. Röber, ``The value of our digital
identity,'' \emph{Boston Cons. Gr}, 2012 {[}Online{]}. Available:
\url{https://www.libertyglobal.com/PDF/public-policy/The-Value-of-Our-Digital-Identity.pdf}

\hypertarget{ref-regulation_2016_eu_general-data-protection-regulation_definition}{}
{[}17{]} \emph{General data protection regulation}. 2016, p. L 119/33
{[}Online{]}. Available:
\url{http://eur-lex.europa.eu/legal-content/EN/TXT/?uri=CELEX:32016R0679}

\hypertarget{ref-web_2016_wikipedia_information-privacy-law_us}{}
{[}18{]} Wikipedia, ``Information privacy law,'' 13-Nov-2016.
{[}Online{]}. Available:
\url{https://en.wikipedia.org/wiki/Information_privacy_law\#United_States}.
{[}Accessed: 20-Nov-2016{]}

\hypertarget{ref-web_2016_data-protection-laws-in-the-us}{}
{[}19{]} I. J. (Loeb \& Loeb), ``PLC - data protection in the united
states: Overview,'' 01-Jul-2013. {[}Online{]}. Available:
\url{http://us.practicallaw.com/6-502-0467}. {[}Accessed: 20-Nov-2016{]}

\hypertarget{ref-web_2015_white-house-releases-consumer-privacy-bill-draft}{}
{[}20{]} A. Wilhelm, ``White house drops `consumer privacy bill of
rights act' draft. TechCrunch,'' 27-Feb-2015. {[}Online{]}. Available:
\url{http://social.techcrunch.com/2015/02/27/white-house-drops-consumer-privacy-bill-of-rights-act-draft/}.
{[}Accessed: 20-Nov-2016{]}

\hypertarget{ref-bill-draft_2015_us_consumer-privacy-bill-of-rights-act_definition}{}
{[}21{]} \emph{Administration discussion draft: Consumer privacy bill of
rights act of 2015}. 2015 {[}Online{]}. Available:
\url{https://www.whitehouse.gov/sites/default/files/omb/legislative/letters/cpbr-act-of-2015-discussion-draft.pdf}

\hypertarget{ref-rules_2016_fcc_to-protect-broadband-consumer-privacy_sensitive-types-of-data}{}
{[}22{]} \emph{Report and order}. 2016 {[}Online{]}. Available:
\url{https://transition.fcc.gov/Daily_Releases/Daily_Business/2016/db1103/FCC-16-148A1.pdf}.
{[}Accessed: 20-Nov-2016{]}

\hypertarget{ref-rules_2016_fcc_to-protect-broadband-consumer-privacy_personally-identifiable-information}{}
{[}23{]} \emph{Notice of proposed rulemaking}. 2016 {[}Online{]}.
Available:
\url{https://apps.fcc.gov/edocs_public/attachmatch/FCC-16-39A1.pdf}.
{[}Accessed: 20-Nov-2016{]}

\hypertarget{ref-web_2016_privacy-policies-are-mandatory-by-law}{}
{[}24{]} ``Privacy policies are mandatory by law,'' 23-Oct-2016.
{[}Online{]}. Available:
\url{https://termsfeed.com/blog/privacy-policy-mandatory-law/}.
{[}Accessed: 20-Nov-2016{]}

\hypertarget{ref-web_2016_international-privacy-standards}{}
{[}25{]} ``International privacy standards,'' 29-Sep-2016. {[}Online{]}.
Available:
\url{https://www.eff.org/issues/international-privacy-standards}.
{[}Accessed: 20-Nov-2016{]}

\hypertarget{ref-paper_2014_who-owns-yours-data}{}
{[}26{]} G. Rosner, ``Who owns your data?'' presented at the UbiComp
'14, september 13 - 17 2014, seattle, wa, usa, 2014, pp. 623--628
{[}Online{]}. Available:
\url{http://dl.acm.org/citation.cfm?doid=2638728.2641679}. {[}Accessed:
01-Dec-2016{]}

\hypertarget{ref-book_1987_private-ownership_definition}{}
{[}27{]} J. Grunebaum, \emph{Private ownership}. Routledge \& Kegan
Paul, 1987, p. 213.

\hypertarget{ref-regulation_2016_eu_general-data-protection-regulation_ownership}{}
{[}28{]} \emph{General data protection regulation}. 2016, p. L 119/12
{[}Online{]}. Available:
\url{http://eur-lex.europa.eu/legal-content/EN/TXT/?uri=CELEX:32016R0679}

\hypertarget{ref-rules_2016_fcc_to-protect-broadband-consumer-privacy_ownership}{}
{[}29{]} \emph{Report and order}. 2016 {[}Online{]}. Available:
\url{https://transition.fcc.gov/Daily_Releases/Daily_Business/2016/db1103/FCC-16-148A1.pdf}.
{[}Accessed: 20-Nov-2016{]}

\hypertarget{ref-web_2016_facebook_terms-of-service}{}
{[}30{]} Facebook, ``Facebooks's terms of service. Statement of rights
and responsibilities,'' 30-Jan-2015. {[}Online{]}. Available:
\url{https://www.facebook.com/legal/terms}. {[}Accessed: 01-Dec-2016{]}

\hypertarget{ref-web_2016_twitter_terms-of-service}{}
{[}31{]} Twitter, ``Twitters's terms of service. Twitter terms of
service,'' 30-Sep-2016. {[}Online{]}. Available:
\url{https://twitter.com/tos\#intlTerms}. {[}Accessed: 01-Dec-2016{]}

\hypertarget{ref-web_2016_google_terms-of-service}{}
{[}32{]} Google, ``Google's terms of service. Google terms of service,''
30-Apr-2014. {[}Online{]}. Available:
\url{https://www.google.com/intl/en/policies/terms/regional.html}.
{[}Accessed: 01-Dec-2016{]}

\hypertarget{ref-web_2016_apple-icloud_terms-of-service}{}
{[}33{]} Apple, ``Apple's iClound terms and conditions. V. content and
your conduct,'' 25-Sep-2016. {[}Online{]}. Available:
\url{https://www.apple.com/legal/internet-services/icloud/en/terms.html}.
{[}Accessed: 01-Dec-2016{]}

\hypertarget{ref-web_2013_why-metadata-matters}{}
{[}34{]} ``Why metadata matters,'' 07-Jun-2013. {[}Online{]}. Available:
\url{https://www.eff.org/deeplinks/2013/06/why-metadata-matters}.
{[}Accessed: 24-Nov-2016{]}

\hypertarget{ref-web_2016_why-you-need-metadata-for-big-data-to-success}{}
{[}35{]} J. P. Stevens, ``Why you need metadata for big data success,''
06-Apr-2016. {[}Online{]}. Available:
\url{http://www.datasciencecentral.com/profiles/blogs/why-you-need-metadata-for-big-data-success}.
{[}Accessed: 24-Nov-2016{]}

\hypertarget{ref-web_2016_oxford_definition_big-data}{}
{[}36{]} ``Big data n.'' {[}Online{]}. Available:
\url{http://www.oed.com/view/Entry/18833\#eid301162177}. {[}Accessed:
11-Nov-2016{]}

\hypertarget{ref-web_2016_wikipedia_definition_big-data}{}
{[}37{]} Wikipedia, ``Big data,'' 11-Nov-2016. {[}Online{]}. Available:
\url{https://en.wikipedia.org/w/index.php?title=Big_data\&oldid=748964100}.
{[}Accessed: 11-Nov-2016{]}

\hypertarget{ref-paper_2015_big-data-analytics_a-survey}{}
{[}38{]} C.-W. Tsai, C.-F. Lai, H.-C. Chao, and A. V. Vasilakos, ``Big
data analytics: A survey,'' \emph{Journal of Big Data}, vol. 2, no. 1,
p. 21, Oct. 2015 {[}Online{]}. Available:
\url{http://journalofbigdata.springeropen.com/articles/10.1186/s40537-015-0030-3}.
{[}Accessed: 13-Nov-2016{]}

\hypertarget{ref-book-chapter_1999_Principles-of-knowledge-discovery-in-databases_introduction-to-data-mining}{}
{[}39{]} O. R. Zaïane, \emph{Principles of knowledge discovery in
databases}. 1999, pp. 1--2 {[}Online{]}. Available:
\url{https://webdocs.cs.ualberta.ca/~zaiane/courses/cmput690/notes/Chapter1/}.
{[}Accessed: 13-Nov-2016{]}

\hypertarget{ref-web_2013_big-data-collection-collides-with-privacy-concerns}{}
{[}40{]} ``Big data collection collides with privacy concerns, analysts
say. PCWorld,'' 10-Feb-2013. {[}Online{]}. Available:
\url{http://www.pcworld.com/article/2027789/big-data-collection-collides-with-privacy-concerns-analysts-say.html}.
{[}Accessed: 15-Nov-2016{]}

\hypertarget{ref-web_2016_answers-io}{}
{[}41{]} ``Answers.io. Answers.'' {[}Online{]}. Available:
\url{https://answers.io/answers}. {[}Accessed: 14-Nov-2016{]}

\hypertarget{ref-web_2016_big-data-enthusiasts-should-not-ignore}{}
{[}42{]} A. L. Burgelman, N. L. Burgelman, and NGDATA, ``Attention, big
data enthusiasts: Here's what you shouldn't ignore. WIRED.''
{[}Online{]}. Available:
\url{https://www.wired.com/insights/2013/02/attention-big-data-enthusiasts-heres-what-you-shouldnt-ignore/}.
{[}Accessed: 15-Nov-2016{]}

\hypertarget{ref-report_2001_3d-data-management-controlling-data-volume-velocity-and-variety}{}
{[}43{]} D. Laney, ``3D data management: Controlling data volume,
velocity, and variety,'' META Group, February 2001 {[}Online{]}.
Available:
\url{http://blogs.gartner.com/doug-laney/files/2012/01/ad949-3D-Data-Management-Controlling-Data-Volume-Velocity-and-Variety.pdf}

\hypertarget{ref-paper_2015_big-data-for-development-a-review-of-promises-and-challenges:more-data}{}
{[}44{]} M. Hilbert, ``Big data for development: A review of promises
and challenges,'' \emph{Development Policy Review}, vol. 34, no. 1, pp.
135--174, December 2015 {[}Online{]}. Available:
\url{http://dx.doi.org/10.1111/dpr.12142}

\hypertarget{ref-web_2016_the-state-of-big-data}{}
{[}45{]} N. Davis Kho, ``The state of big data,'' 24-Feb-2016.
{[}Online{]}. Available:
\url{http://www.econtentmag.com/Articles/Editorial/Feature/The-State-of-Big-Data-108666.htm}.
{[}Accessed: 18-Nov-2016{]}

\hypertarget{ref-web_2016_apple_customer-letter}{}
{[}46{]} T. C. (Apple's CEO), ``A message to our customers. Customer
letter,'' 16-Feb-2016. {[}Online{]}. Available:
\url{http://www.apple.com/customer-letter/}. {[}Accessed: 18-Nov-2016{]}

\hypertarget{ref-web_2016_what-is-differential-privacy}{}
{[}47{]} M. Green, ``What is differential privacy? A few thoughts on
cryptographic engineering,'' 15-Jun-2016. {[}Online{]}. Available:
\url{https://blog.cryptographyengineering.com/2016/06/15/what-is-differential-privacy/}.
{[}Accessed: 18-Nov-2016{]}

\hypertarget{ref-web_2016_eff_whatsapp-rolls-out-emd-to-end-encryption}{}
{[}48{]} B. Budington, ``WhatsApp rolls out end-to-end encryption to its
over one billion users,'' 07-Apr-2016. {[}Online{]}. Available:
\url{https://www.eff.org/deeplinks/2016/04/whatsapp-rolls-out-end-end-encryption-its-1bn-users}.
{[}Accessed: 18-Nov-2016{]}

\hypertarget{ref-web_2007_introducing-google-traffic}{}
{[}49{]} ``Stuck in traffic? Insights from googlers into our products,
technology, and the google culture,'' 28-Feb-2007. {[}Online{]}.
Available:
\url{https://googleblog.blogspot.com/2007/02/stuck-in-traffic.html}.
{[}Accessed: 18-Nov-2016{]}

\hypertarget{ref-web_2016_wikipedia_google-traffic}{}
{[}50{]} Wikipedia, ``Google traffic,'' 25-Oct-2016. {[}Online{]}.
Available:
\url{https://en.wikipedia.org/w/index.php?title=Google_Traffic\&oldid=746200591}.
{[}Accessed: 18-Nov-2016{]}

\hypertarget{ref-graphic_2016_global-mobile-os-market-share}{}
{[}51{]} ``Global mobile OS market share.'' {[}Online{]}. Available:
\url{https://www.statista.com/statistics/266136/global-market-share-held-by-smartphone-operating-systems/}.
{[}Accessed: 18-Nov-2016{]}

\hypertarget{ref-estimating-the-locations-of-emergency-events-from-twitter-streams_2014}{}
{[}52{]} J. Ao, P. Zhang, and Y. Cao, ``Estimating the Locations of
Emergency Events from Twitter Streams,'' \emph{Procedia Computer
Science}, vol. 31, pp. 731--739, 2014 {[}Online{]}. Available:
\url{http://linkinghub.elsevier.com/retrieve/pii/S1877050914004980}.
{[}Accessed: 05-Nov-2016{]}

\hypertarget{ref-the-practice-of-predictive-analytics-in-healthcare_2013}{}
{[}53{]} G. Palem, ``The Practice of Predictive Analytics in
Healthcare,'' \emph{ResearchGate}, Apr. 2013 {[}Online{]}. Available:
\url{https://www.researchgate.net/publication/236336250_The_Practice_of_Predictive_Analytics_in_Healthcare}.
{[}Accessed: 05-Nov-2016{]}

\hypertarget{ref-data-collection-for-climate-changes_2014}{}
{[}54{]} N. Burger, B. Ghosh-Dastidar, A. Grant, G. Joseph, T. Ruder, O.
Tchakeva, and Q. Wodon, ``Data Collection for the Study on Climate
Change and Migration in the MENA Region,'' 2014 {[}Online{]}. Available:
\url{https://mpra.ub.uni-muenchen.de/56929/}. {[}Accessed:
04-Nov-2016{]}

\hypertarget{ref-graphic_2015_applications-of-big-data-in-10-industry-verticals}{}
{[}55{]} M. Gaitho, ``Applications of big data in 10 industry
verticals,'' 20-Oct-2015. {[}Online{]}. Available:
\url{https://www.simplilearn.com/big-data-applications-in-industries-article}.
{[}Accessed: 19-Nov-2016{]}

\hypertarget{ref-graphic_2012_personal-data-ecosystem}{}
{[}56{]} F. T. C. USA, ``Personal data ecosystem,'' \emph{Protecting
Consumer Privacy in an Era of Rapid Change - Recommendations for
Business and Policymakers - FTC Report}, March-2012. {[}Online{]}.
Available:
\url{https://www.ftc.gov/sites/default/files/documents/public_events/exploring-privacy-roundtable-series/personaldataecosystem.pdf}.
{[}Accessed: 17-Nov-2016{]}

\hypertarget{ref-paper_1965_moors-law}{}
{[}57{]} G. E. Moore, ``Cramming more components onto integrated
circuits,'' \emph{Electronics}, vol. 38, p. 4, Apr. 1965 {[}Online{]}.
Available:
\url{https://drive.google.com/file/d/0By83v5TWkGjvQkpBcXJKT1I1TTA/}.
{[}Accessed: 07-Dec-2016{]}

\hypertarget{ref-podcast_2015_cre-neuronale-netze}{}
{[}58{]} T. Pritlove and U. Schöneberg, \emph{Neuronale netze}. 2015
{[}Online{]}. Available: \url{https://cre.fm/cre208-neuronale-netze}.
{[}Accessed: 06-Dec-2016{]}

\hypertarget{ref-web_2016_industries-intention-to-invest-in-big-data}{}
{[}59{]} L. Columbus, ``51\% of enterprises intend to invest more in big
data,'' 22-May-2016. {[}Online{]}. Available:
\url{http://www.forbes.com/sites/louiscolumbus/2016/05/22/51-of-enterprises-intend-to-invest-more-in-big-data/}.
{[}Accessed: 07-Dec-2016{]}

\hypertarget{ref-web_2016_projectvrm_development-work}{}
{[}60{]} ``ProjectVRM - cDevelopment work. ProjectVRM,'' 28-Nov-2016.
{[}Online{]}. Available:
\url{https://cyber.harvard.edu/projectvrm/VRM_Development_Work}.
{[}Accessed: 09-Dec-2016{]}

\hypertarget{ref-web_2016_projectvrm_principles}{}
{[}61{]} ``ProjectVRM - principles. ProjectVRM,'' 28-Nov-2016.
{[}Online{]}. Available:
\url{https://cyber.harvard.edu/projectvrm/Main_Page\#VRM_Principles}.
{[}Accessed: 09-Dec-2016{]}

\hypertarget{ref-graphic_2011_architecture_components-of-organization-domain}{}
{[}62{]} The TAS3 Consortium, ``TAS3 architecture - figure 2.2: Major
components of organization domain.'' Jul. 2011 {[}Online{]}. Available:
\url{http://homes.esat.kuleuven.ac.be/~decockd/tas3/final.deliverables/pm42/TAS3_D02p1_TAS3.Architecture_final.pdf}

\hypertarget{ref-web_kantara-initiative}{}
{[}63{]} ``Kantara initiative -- join. innovate. trust.'' {[}Online{]}.
Available: \url{https://kantarainitiative.org/}. {[}Accessed:
14-Dec-2016{]}

\hypertarget{ref-paper_2014_personal-data-store-approach}{}
{[}64{]} T. Kirkham, S. Winfield, S. Ravet, and S. Kellomaki, ``The
personal data store approach to personal data security,'' \emph{IEEE
Security \& Privacy}, vol. 11, no. 5, pp. 12--19, 2013.

\hypertarget{ref-paper_2012_openpds_on-trusted-use-of-large-scale-personal-data}{}
{[}65{]} Y.-A. de Montjoye, S. S. Wang, A. Pentland, D. T. T. Anh, A.
Datta, and others, ``On the trusted use of large-scale personal data.''
\emph{IEEE Data Eng. Bull.}, vol. 35, no. 4, pp. 5--8, 2012
{[}Online{]}. Available:
\url{http://sites.computer.org/debull/a12dec/a12dec-cd.pdf\#page=7}.
{[}Accessed: 30-Oct-2016{]}

\hypertarget{ref-web_mit_openpds-safeanswers-project-page}{}
{[}66{]} ``openPDS/SafeAnswers - the privacy-preserving personal data
store.'' {[}Online{]}. Available: \url{http://openpds.media.mit.edu/}.
{[}Accessed: 14-Dec-2016{]}

\hypertarget{ref-paper_2014_openpds_protecting-privacy-of-meta-data-through-safeanswers}{}
{[}67{]} Y.-A. de Montjoye, E. Shmueli, S. S. Wang, and A. S. Pentland,
``openPDS: Protecting the privacy of metadata through SafeAnswers,''
\emph{PLoS ONE}, vol. 9, no. 7, p. e98790, Jul. 2014 {[}Online{]}.
Available: \url{http://dx.plos.org/10.1371/journal.pone.0098790}.
{[}Accessed: 30-Oct-2016{]}

\hypertarget{ref-web_microsoft_healthvault}{}
{[}68{]} ``Microsoft HealthVault. Overview.'' {[}Online{]}. Available:
\url{https://www.healthvault.com/de/en/overview}. {[}Accessed:
14-Dec-2016{]}

\hypertarget{ref-web_meeco_how-it-works}{}
{[}69{]} ``How it works meeco.'' {[}Online{]}. Available:
\url{https://meeco.me/how-it-works.html}. {[}Accessed: 14-Dec-2016{]}

\hypertarget{ref-slides_2015_meeco-case-study}{}
{[}70{]} M. Page, ``Online adver\textgreater{}sing -- booming or
broken?'' Sep-2015 {[}Online{]}. Available:
\url{https://meeco.me/assets/pdf/Meeco_Case_Study_Online_Advertising-Booming_or_Broken_Sept_2015.pdf}

\hypertarget{ref-web_industrial-data-space}{}
{[}71{]} ``The principles. Industrial data space e.V.'' {[}Online{]}.
Available: \url{http://www.industrialdataspace.org/en/the-principles/}.
{[}Accessed: 14-Dec-2016{]}

\hypertarget{ref-whitepaper_2016_industrial-data-space}{}
{[}72{]} B. Prof. Dr.-Ing. Otto, S. Prof. Dr. Auer, J. Cirullies, J.
Prof. Dr. Jürjens, N. Menz, J. Schon, and S. Dr. Wenzel, ``Industrial
data space - digital sovereignity over data.'' Fraunhofer-Gesellschaft
zur Förderung der angewandten Forschung e.V., 17-Aug-2016 {[}Online{]}.
Available:
\url{http://www.industrialdataspace.org/wp-content/uploads/2016/09/whitepaper-industrial-data-space-eng.pdf}

\hypertarget{ref-web_spec_http1}{}
{[}73{]} P. J. Leach, T. Berners-Lee, J. C. Mogul, L. Masinter, R. T.
Fielding, and J. Gettys, ``Hypertext transfer protocol -- HTTP/1.1,''
Jun-1999. {[}Online{]}. Available:
\url{https://tools.ietf.org/html/rfc2616}. {[}Accessed: 17-Dec-2016{]}

\hypertarget{ref-web_spec_http2}{}
{[}74{]} M. Belshe, M. Thomson, and R. Peon, ``Hypertext transfer
protocol version 2 (HTTP/2),'' May-2015. {[}Online{]}. Available:
\url{https://tools.ietf.org/html/rfc7540}. {[}Accessed: 17-Dec-2016{]}

\hypertarget{ref-web_spec_tls}{}
{[}75{]} T. Dierks and E. Rescorla, ``The transport layer security (TLS)
protocol version 1.2,'' Aug-2008. {[}Online{]}. Available:
\url{https://tools.ietf.org/html/rfc5246}. {[}Accessed: 17-Dec-2016{]}

\hypertarget{ref-web_spec_websockets}{}
{[}76{]} I. Fette and A. Melnikov, ``The WebSocket protocol,'' Dec-2011.
{[}Online{]}. Available: \url{https://tools.ietf.org/html/rfc6455}.
{[}Accessed: 17-Dec-2016{]}

\hypertarget{ref-web_spec_json}{}
{[}77{]} D. Crockford, ``The JSON data interchange format.'' ECMA
International, Oct-2013 {[}Online{]}. Available:
\url{http://www.ecma-international.org/publications/files/ECMA-ST/ECMA-404.pdf}

\hypertarget{ref-web_rfc_json}{}
{[}78{]} T. Bray, ``The JavaScript object notation (JSON) data
interchange format,'' Mar-2014. {[}Online{]}. Available:
\url{https://tools.ietf.org/html/rfc7159}. {[}Accessed: 17-Dec-2016{]}

\hypertarget{ref-web_2012_problem-with-oauth-for-authentication}{}
{[}79{]} J. Bradley, ``The problem with OAuth for authentication.''
28-Jan-2012. {[}Online{]}. Available:
\url{http://www.thread-safe.com/2012/01/problem-with-oauth-for-authentication.html}.
{[}Accessed: 17-Dec-2016{]}

\hypertarget{ref-web_spec_oauth-1a}{}
{[}80{]} ``OAuth core 1.0a.'' {[}Online{]}. Available:
\url{https://oauth.net/core/1.0a/}. {[}Accessed: 18-Dec-2016{]}

\hypertarget{ref-web_spec_oauth-2}{}
{[}81{]} D. Hardt, ``The OAuth 2.0 authorization framework,'' Oct-2012.
{[}Online{]}. Available: \url{https://tools.ietf.org/html/rfc6749}.
{[}Accessed: 18-Dec-2016{]}

\hypertarget{ref-web_2016_oauth-2}{}
{[}82{]} I. O. WG, ``OAuth 2.0.'' {[}Online{]}. Available:
\url{https://oauth.net/2/}. {[}Accessed: 16-Dec-2016{]}

\hypertarget{ref-web_spec_openid-connect-1}{}
{[}83{]} ``OpenID connect core 1.0 incorporating errata set 1,''
08-Nov-2014. {[}Online{]}. Available:
\url{https://openid.net/specs/openid-connect-core-1_0.html}.
{[}Accessed: 17-Dec-2016{]}

\hypertarget{ref-web_spec_json-web-token}{}
{[}84{]} J. Bradley, N. Sakimura, and M. Jones, ``JSON web token
(JWT),'' May-2015. {[}Online{]}. Available:
\url{https://tools.ietf.org/html/rfc7519}. {[}Accessed: 17-Dec-2016{]}

\hypertarget{ref-web_spec_json-web-encryption}{}
{[}85{]} J. Hildebrand and M. Jones, ``JSON web encryption (JWE),''
May-2015. {[}Online{]}. Available:
\url{https://tools.ietf.org/html/rfc7516}. {[}Accessed: 17-Dec-2016{]}

\hypertarget{ref-web_spec_json-web-signature}{}
{[}86{]} J. Bradley, N. Sakimura, and M. Jones, ``JSON web signature
(JWS),'' May-2015. {[}Online{]}. Available:
\url{https://tools.ietf.org/html/rfc7515}. {[}Accessed: 17-Dec-2016{]}

\hypertarget{ref-web_spec_rest}{}
{[}87{]} T. Fielding, ``Representational state transfer (REST),'' in
\emph{Architectural styles and the design of network-based software
architectures}, University of California, Irvine, 2000, pp. 76--106
{[}Online{]}. Available:
\url{https://www.ics.uci.edu/~fielding/pubs/dissertation/fielding_dissertation.pdf}

\hypertarget{ref-web_spec_http-methods}{}
{[}88{]} P. J. Leach, T. Berners-Lee, J. C. Mogul, L. Masinter, R. T.
Fielding, and J. Gettys, ``HTTP methods,'' Jun-1999. {[}Online{]}.
Available: \url{https://tools.ietf.org/html/rfc2616\#section-9}.
{[}Accessed: 18-Dec-2016{]}

\hypertarget{ref-web_spec_graphql}{}
{[}89{]} ``GraphQL,'' Oct-2016. {[}Online{]}. Available:
\url{https://facebook.github.io/graphql/}. {[}Accessed: 17-Dec-2016{]}

\hypertarget{ref-web_w3c-tr_rdf}{}
{[}90{]} D. Beckett and B. McBride, ``RDF/XML syntax specification
(revised),'' 10-Feb-2004. {[}Online{]}. Available:
\url{https://www.w3.org/TR/REC-rdf-syntax/}. {[}Accessed: 19-Dec-2016{]}

\hypertarget{ref-web_w3c-tr_owl}{}
{[}91{]} W. O. W. Group, ``OWL 2 web ontology language document overview
(second edition),'' 11-Dec-2012. {[}Online{]}. Available:
\url{https://www.w3.org/TR/owl2-overview/}. {[}Accessed: 19-Dec-2016{]}

\hypertarget{ref-web_w3c-tr_sparql}{}
{[}92{]} S. Harris, A. Seaborne, and E. Prud'hommeaux, ``SPARQL 1.1
query language,'' 21-Mar-2013. {[}Online{]}. Available:
\url{https://www.w3.org/TR/sparql11-query/}. {[}Accessed: 19-Dec-2016{]}

\hypertarget{ref-web_w3c-draft_webid}{}
{[}93{]} ``WebID specifications.'' {[}Online{]}. Available:
\url{https://www.w3.org/2005/Incubator/webid/spec/}. {[}Accessed:
19-Dec-2016{]}

\hypertarget{ref-web_spec_solid}{}
{[}94{]} ``Solid specification,'' 03-Mar-2016. {[}Online{]}. Available:
\url{https://github.com/solid/solid-spec}. {[}Accessed: 17-Dec-2016{]}

\hypertarget{ref-web_2016_wiki_webaccesscontrol}{}
{[}95{]} ``WebAccessControl - w3c wiki.'' {[}Online{]}. Available:
\url{https://www.w3.org/wiki/WebAccessControl}. {[}Accessed:
19-Dec-2016{]}

\hypertarget{ref-web_2016_demo_databox}{}
{[}96{]} ``Databox.me.'' {[}Online{]}. Available:
\url{https://databox.me/}. {[}Accessed: 19-Dec-2016{]}

\hypertarget{ref-web_2015_cgroup-doc}{}
{[}97{]} T. Heo, ``Control group (v2) documentation,'' Oct-2015.
{[}Online{]}. Available:
\url{https://www.kernel.org/doc/Documentation/cgroup-v2.txt}.
{[}Accessed: 20-Dec-2016{]}

\hypertarget{ref-web_2016_kernel-namespace}{}
{[}98{]} ``Overview of linux namespaces,'' 12-Dec-2016. {[}Online{]}.
Available: \url{http://man7.org/linux/man-pages/man7/namespaces.7.html}.
{[}Accessed: 20-Dec-2016{]}

\hypertarget{ref-web_2016_open-container-initiative}{}
{[}99{]} ``Open container initiative.'' {[}Online{]}. Available:
\url{https://www.opencontainers.org/}. {[}Accessed: 20-Dec-2016{]}

\hypertarget{ref-web_oci-spec_runtime}{}
{[}100{]} ``Container runtime specification (v1.0.0-rc3),'' 12-Dec-2016.
{[}Online{]}. Available:
\url{https://github.com/opencontainers/runtime-spec/tree/v1.0.0-rc3}.
{[}Accessed: 20-Dec-2016{]}

\hypertarget{ref-web_oci-spec_image}{}
{[}101{]} ``Container image specification (v1.0.0-rc3),'' 30-Nov-2016.
{[}Online{]}. Available:
\url{https://github.com/opencontainers/image-spec/tree/v1.0.0-rc3}.
{[}Accessed: 20-Dec-2016{]}

\hypertarget{ref-web_2009-success-of-facebook-connect}{}
{[}102{]} N. Carlson, ``Facebook connect is a huge success -- by the
numbers,'' 01-Jul-2009. {[}Online{]}. Available:
\url{http://www.businessinsider.com/six-months-in-facebook-connect-is-a-huge-success-2009-7}.
{[}Accessed: 16-Dec-2016{]}

\end{document}
